\documentclass[12pt,letterpaper]{article}
\usepackage[margin=1in]{geometry}
\usepackage{fancyhdr}
\usepackage[utf8]{inputenc}
\usepackage{palatino}
\usepackage{microtype}
\usepackage{hyperref}
\usepackage{graphicx}
\usepackage{lastpage}
\usepackage[hang,small]{caption}
\usepackage{titlesec}
\usepackage{amsmath,amssymb}
\usepackage{multirow}

\renewcommand{\headrulewidth}{0pt}
\fancyfoot{}
\fancyfoot[C]{\sf Page \thepage\ of \pageref{LastPage}}
\pagestyle{fancy}

\titleformat{\section}{\bfseries\Large}{\arabic{\thesection}}{1em}{}
\titleformat{\subsection}{\bfseries\large}{\arabic{\thesection}.\arabic{\thesubsection}}{1em}{}
\titleformat{\subsubsection}{\itshape}{\arabic{\thesection}.\arabic{\thesubsection}.\arabic{\thesubsubsection}}{1em}{}

\setlength{\parindent}{0cm}
\setlength{\parskip}{0.8em}

\captionsetup[figure]{labelfont=it,font=it}
\captionsetup[table]{labelfont={it,sc},font={it,sc}}

\hypersetup{colorlinks,
    linkcolor = black,
    citecolor = black,
    urlcolor  = black}
\urlstyle{same}



\begin{document}

Soo-Hyun Yoo \\
CS472 \\
Homework 3 \\
November 7, 2014

\section*{Number Representations}

\begin{enumerate}
    \item[3.1.1]
        \begin{enumerate}
            \item $3716$
            \item $6041$
        \end{enumerate}
    \item[3.1.2]
        \begin{enumerate}
            \item $3716$
            \item $1654 - 0165 = 1467$
        \end{enumerate}
    \item[3.1.3]
        \begin{enumerate}
            \item Unsigned: $3\cdot8^3+1\cdot8^2+7\cdot8+4 = \boxed{1660}$  Signed: $\boxed{1660}$
            \item Unsigned: $4\cdot8^3+1\cdot8^2+6\cdot8+5 = \boxed{2165}$  Signed: $-(1\cdot8^2+6\cdot8+5) = \boxed{-117}$
        \end{enumerate}
    \item[3.1.4]
        \begin{enumerate}
            \item $6374$
            \item $753$
        \end{enumerate}
    \item[3.1.5]
        \begin{enumerate}
            \item $-3040 - 444 = \boxed{-3504}$ ($7504$ in sign-mag format)
            \item $-365 - 3412 = \boxed{-3777}$ ($7777$ in sign-mag format)
        \end{enumerate}
    \item[3.1.6]
        \begin{enumerate}
            \item $111000100000$
            \item $100011110101$

                Octal is a convenient way to condense binary numbers because
                8 is a power of 2, although there is little to be gained by
                using octal over hexadecimal.
        \end{enumerate}
\end{enumerate}


\section*{Floating Point}

Accounting for the implied 1 of the significand and the exponent bias of 127:

$14.5 = 0 \; 10000010 \; 11010000000000000000000$

$35.25 = 0 \; 10000100 \; 00011010000000000000000$

$14.5 + 35.25 = 1110.1 + 100011.01 = 110001.11$, so

$14.5 + 35.25 = 0 \; 10000100 \; 100011100000000000000$


\section*{Hardware Blocks}

\begin{enumerate}
    \item[4.1.1]
        \begin{enumerate}
            \item
            \item
        \end{enumerate}
    \item[4.1.2]
        \begin{enumerate}
            \item PC, Instruction Memory, Registers, Mux, ALU, PC adder
            \item PC, Instruction Memory, Registers, Mux, ALU, PC adder, Data Memory
        \end{enumerate}
    \item[4.1.3]
        \begin{enumerate}
            \item Used: Instruction, Registers, ALU, Mux, PC adder, RegWrite, ALU Operation, Branch, Zero

                Unused: MemRead, MemWrite
            \item Used: Instruction, Registers, ALU, Mux, PC adder, RegWrite, ALU Operation, Branch, Zero, MemRead, MemWrite

                Unused: None
        \end{enumerate}
    \item[4.1.4]
        \begin{enumerate}
            \item Imem - Regs - Mux - ALU - Mux - Regs
            \item Imem - Regs - Mux - ALU - Mux - Regs
        \end{enumerate}
    \item[4.1.5]
        \begin{enumerate}
            \item Imem - Control - Dmem - Mux - Regs
            \item Imem - Control - Dmem - Mux - Regs
        \end{enumerate}
    \item[4.1.6]
        \begin{enumerate}
            \item Imem - Regs - Mux - ALU - Mux
            \item Imem - Regs - Mux - ALU - Mux
        \end{enumerate}
\end{enumerate}

\end{document}

