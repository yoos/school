\documentclass[12pt,letterpaper]{article}
\usepackage[margin=1in]{geometry}
\usepackage{fancyhdr}
\usepackage[utf8]{inputenc}
\usepackage{palatino}
\usepackage{microtype}
\usepackage{hyperref}
\usepackage{graphicx}
\usepackage{lastpage}
\usepackage[hang,small]{caption}
\usepackage{titlesec}
\usepackage{amsmath,amssymb}
\usepackage{multirow}

\renewcommand{\headrulewidth}{0pt}
\fancyfoot{}
\fancyfoot[C]{\sf Page \thepage\ of \pageref{LastPage}}
\pagestyle{fancy}

\titleformat{\section}{\bfseries\Large}{\arabic{\thesection}}{1em}{}
\titleformat{\subsection}{\bfseries\large}{\arabic{\thesection}.\arabic{\thesubsection}}{1em}{}
\titleformat{\subsubsection}{\itshape}{\arabic{\thesection}.\arabic{\thesubsection}.\arabic{\thesubsubsection}}{1em}{}

\setlength{\parindent}{0cm}
\setlength{\parskip}{0.8em}

\captionsetup[figure]{labelfont=it,font=it}
\captionsetup[table]{labelfont={it,sc},font={it,sc}}

\hypersetup{colorlinks,
    linkcolor = black,
    citecolor = black,
    urlcolor  = black}
\urlstyle{same}



\begin{document}

Soo-Hyun Yoo \\
CS472 \\
Lab 1\\
October 10, 2014

The performance of addition, multiplication, and division operations is
examined for floating point and integers on a 8-bit ATmega328p and a 64-bit
Intel Core i7-3612QM.

\section*{Setup}

\subsection*{8-bit Microcontroller}

An ATmega328p clocked at 16 MHz was set up with code such as the following to
toggle a GPIO pin after every operation.

\begin{verbatim}
void main(void) {
    DDRB |= _BV(5);

    float a = 0.0;
    float c = 99999.645543999993;
    float b = 999.99999999999999;

    while(1) {
        PORTB = ~(PORTB & _BV(5));
        a = b + c;
    }
}
\end{verbatim}

The frequency $f$ of the resulting square wave was measured with an
oscilloscope. The number of operations per second was then $2f$.

\subsection*{64-bit Laptop}

A laptop with a 64-bit Intel Core i7-3612QM was set up with code such as the
following to run 10 billion operations using a for loop. It is worth noting
that the i7 throttles up from an idle state of 1.2 GHz to 3.0 GHz when running
the benchmark.

\begin{verbatim}
void main(void)
{
    float a=0;

    long long int i;
    for (i=0; i<10000000000; i++) {
        a /= 1.0f;
    }
}
\end{verbatim}

The UNIX {\tt time} tool was used to measure the runtime $t$ of the program.
The number of operations per second was then $\cfrac{10,000,000,000}{t}$.


\section*{Results and Analysis}

The two devices perform the following number of operations per second:

\begin{table}[!h]
	\centering
	\caption{Operations per second}
	\begin{tabular}{|l|c|c|} \hline
		          & ATmega328p & i7-3612QM \\ \hline\hline
		Integer + & 571430 & 425985090 \\ \hline
		Integer x & 454545 & 426257459 \\ \hline
		Integer / & 65789  & 119631534 \\ \hline
		Float +   & 90909  & 330939537 \\ \hline
		Float x   & 100000 & 270343336 \\ \hline
		Float /   & 30488  & 175561797 \\ \hline
	\end{tabular}
\end{table}

This translates to the following CPI numbers:

\begin{table}[!h]
	\centering
	\caption{Clock cycles per instruction}
	\begin{tabular}{|l|c|c|} \hline
		          & ATmega328p & i7-3612QM \\ \hline\hline
		Integer + & 28  & 7  \\ \hline
		Integer x & 35  & 7  \\ \hline
		Integer / & 243 & 25 \\ \hline
		Float +   & 176 & 9  \\ \hline
		Float x   & 160 & 11 \\ \hline
		Float /   & 525 & 17 \\ \hline
	\end{tabular}
\end{table}

The i7 is nearly an order of magnitude more effective than the ATmega in terms
of CPI for integer operations and half an order of magnitude greater still for
floating point operations.

Division is slow across the board. Division on both devices is slower than any
of the tested floating point operations. The CPI improvement from the ATmega to
the i7 is likely due to architectural optimizations made on the latter.

% x86_64: 10 billion ops:

% floating point
% 	+: 30.217
% 	x: 36.99
% 	/: 56.96
% integer
% 	+: 23.475
% 	x: 23.460
% 	/: 83.59

% atmega:

% fp+: 11 us      90909
% fpx: 10 us      100000
% fp/: 32.8 us    30488
% i+: 1.75 us     571430
% ix: 2.2 us      454545
% i/: 15.2 usA    65789
% fp sin: 115 us
% fp cos: 121 us
% fp tan: 159 us
% fp sqrt: 32 us
% fp powf: 340 us

\end{document}

