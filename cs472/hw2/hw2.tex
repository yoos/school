\documentclass[12pt,letterpaper]{article}
\usepackage[margin=1in]{geometry}
\usepackage{fancyhdr}
\usepackage[utf8]{inputenc}
\usepackage{palatino}
\usepackage{microtype}
\usepackage{hyperref}
\usepackage{graphicx}
\usepackage{lastpage}
\usepackage[hang,small]{caption}
\usepackage{titlesec}
\usepackage{amsmath,amssymb}
\usepackage{multirow}

\renewcommand{\headrulewidth}{0pt}
\fancyfoot{}
\fancyfoot[C]{\sf Page \thepage\ of \pageref{LastPage}}
\pagestyle{fancy}

\titleformat{\section}{\bfseries\Large}{\arabic{\thesection}}{1em}{}
\titleformat{\subsection}{\bfseries\large}{\arabic{\thesection}.\arabic{\thesubsection}}{1em}{}
\titleformat{\subsubsection}{\itshape}{\arabic{\thesection}.\arabic{\thesubsection}.\arabic{\thesubsubsection}}{1em}{}

\setlength{\parindent}{0cm}
\setlength{\parskip}{0.8em}

\captionsetup[figure]{labelfont=it,font=it}
\captionsetup[table]{labelfont={it,sc},font={it,sc}}

\hypersetup{colorlinks,
    linkcolor = black,
    citecolor = black,
    urlcolor  = black}
\urlstyle{same}



\begin{document}

Soo-Hyun Yoo \\
CS472 \\
Homework 2 \\
October 24, 2014

\begin{enumerate}
	\item[2.1.1]
		\begin{enumerate}
			\item {\tt sub f, g, h}
			\item {\tt addi f, h, -5 \\ add f, f, g}
		\end{enumerate}

	\item[2.1.2]
		\begin{enumerate}
			\item 1
			\item 2
		\end{enumerate}

	\item[2.1.3]
		\begin{enumerate}
			\item -1
			\item 0
		\end{enumerate}

	\item[2.1.4]
		\begin{enumerate}
			\item {\tt f = f + 4;}
			\item {\tt f = i + (g + h);}
		\end{enumerate}

	\item[2.1.5]
		\begin{enumerate}
			\item 5
			\item 9
		\end{enumerate}

	\item[2.3.1]
		\begin{enumerate}
			\item {\tt sub f, 0, f \\ sub f, f, g}
			\item {\tt sub f, 0, f \\ addi f, f, -5 \\ add f, f, g}
		\end{enumerate}

	\item[2.3.2]
		\begin{enumerate}
			\item 2
			\item 3
		\end{enumerate}

	\item[2.3.3]
		\begin{enumerate}
			\item -3
			\item -4
		\end{enumerate}

	\item[2.3.4]
		\begin{enumerate}
			\item {\tt f = f - 4;}
			\item {\tt f = g + h;}
		\end{enumerate}

	\item[2.3.1]
		\begin{enumerate}
			\item -3
			\item 5
		\end{enumerate}

	\item[2.7.1]
		\begin{enumerate}
			\item 613566756
			\item 1606303744
		\end{enumerate}

	\item[2.7.2]
		\begin{enumerate}
			\item 613566756
			\item 1606303744
		\end{enumerate}

	\item[2.7.3]
		\begin{enumerate}
			\item 24924924
			\item 5fbe4000
		\end{enumerate}

	\item[2.7.4]
		\begin{enumerate}
			\item 1111 1111 1111 1111
			\item 0000 0100 0000 0000
		\end{enumerate}

	\item[2.7.5]
		\begin{enumerate}
			\item ffff
			\item 0400
		\end{enumerate}

	\item[2.7.6]
		\begin{enumerate}
			\item 0001
			\item fc00
		\end{enumerate}

	\item[2.27.1]
		\begin{enumerate}
			\item {\tt lw \$s1, 20(\$s2)} sets {\tt \$s1 = Memory[\$s2+20]}.
			\item {\tt j 2500} jumps to the target address, 2500.
		\end{enumerate}

	\item[2.27.2]
		\begin{enumerate}
			\item I-type
			\item J-type
		\end{enumerate}

	\item[2.27.4]
		\begin{enumerate}
			\item {\tt 0x00400000: 000100 00001 00010 0000110001000000 \\
				0x00403100: 001000 00001 00001 0000000000000001}
			\item {\tt 0x00000100: 000010 01000000000000000000000100 \\
				0x04000010: 001000 00001 00001 0000000000000001}
		\end{enumerate}

	% \item[2.34]
	% \item[2.35]
\end{enumerate}

\end{document}

