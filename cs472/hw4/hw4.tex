\documentclass[12pt,letterpaper]{article}
\usepackage[margin=1in]{geometry}
\usepackage{fancyhdr}
\usepackage[utf8]{inputenc}
\usepackage{palatino}
\usepackage{microtype}
\usepackage{hyperref}
\usepackage{graphicx}
\usepackage{lastpage}
\usepackage[hang,small]{caption}
\usepackage{titlesec}
\usepackage{amsmath,amssymb}
\usepackage{multirow}

\renewcommand{\headrulewidth}{0pt}
\fancyfoot{}
\fancyfoot[C]{\sf Page \thepage\ of \pageref{LastPage}}
\pagestyle{fancy}

\titleformat{\section}{\bfseries\Large}{\arabic{\thesection}}{1em}{}
\titleformat{\subsection}{\bfseries\large}{\arabic{\thesection}.\arabic{\thesubsection}}{1em}{}
\titleformat{\subsubsection}{\itshape}{\arabic{\thesection}.\arabic{\thesubsection}.\arabic{\thesubsubsection}}{1em}{}

\setlength{\parindent}{0cm}
\setlength{\parskip}{0.8em}

\captionsetup[figure]{labelfont=it,font=it}
\captionsetup[table]{labelfont={it,sc},font={it,sc}}

\hypersetup{colorlinks,
    linkcolor = black,
    citecolor = black,
    urlcolor  = black}
\urlstyle{same}



\begin{document}

Soo-Hyun Yoo \\
CS472 \\
Homework 4 \\
November 26, 2014

\section*{Memory Locality}

\begin{enumerate}
    \item[5.2.1]
        \begin{enumerate}
            \item 8
            \item 8
        \end{enumerate}
    \item[5.2.2]
        \begin{enumerate}
            \item I, J
            \item I, J
        \end{enumerate}
    \item[5.2.3]
        \begin{enumerate}
            \item A, B
            \item A, B
        \end{enumerate}
    \item[5.2.4]
        \begin{enumerate}
            \item 128000
            \item 128000
        \end{enumerate}
    \item[5.2.5]
        \begin{enumerate}
            \item I, J
            \item I, J
        \end{enumerate}
    \item[5.2.6]
        \begin{enumerate}
            \item A, B
            \item A, B
        \end{enumerate}
\end{enumerate}


\section*{Cache Table}

\begin{enumerate}
    \item[5.10.2] Larger page sizes can help reduce the number of TLB misses by
        allowing a TLB of a given size to track more memory. However, it will
        be difficult to pack process memory into the page table if the page
        size is comparatively large, resulting in more wasted memory.
    \item[5.10.3] Virtual memory access without a TLB will have to scan the
        page table directly.
    \item[5.10.4]
        \begin{enumerate}
            \item $\cfrac{2^{32-13} \cdot 4}{2} = \boxed{1\text{ MB}}$
            \item $\cfrac{2^{64-13} \cdot 6}{2} = \boxed{6\text{ PB}}$
        \end{enumerate}
    \item[5.12.5] There are many factors that affect cache performance,
        including read/write ratio, their time cost, amount of cache available,
        and average access size. It is difficult to determine a single policy
        that can address all combinations of such factors, so policies must be
        developed on a more case-by-case basis.
    \item[5.12.6] Having control of the cache decision and knowledge of the
        process being run, I could minimize the miss rate by caching exactly
        the addresses I know will be used most frequently by the process.
\end{enumerate}

\end{document}

