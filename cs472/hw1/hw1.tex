\documentclass[12pt,letterpaper]{article}
\usepackage[margin=1in]{geometry}
\usepackage{fancyhdr}
\usepackage[utf8]{inputenc}
\usepackage{palatino}
\usepackage{microtype}
\usepackage{hyperref}
\usepackage{graphicx}
\usepackage{lastpage}
\usepackage[hang,small]{caption}
\usepackage{titlesec}
\usepackage{amsmath,amssymb}
\usepackage{multirow}

\renewcommand{\headrulewidth}{0pt}
\fancyfoot{}
\fancyfoot[C]{\sf Page \thepage\ of \pageref{LastPage}}
\pagestyle{fancy}

\titleformat{\section}{\bfseries\Large}{\arabic{\thesection}}{1em}{}
\titleformat{\subsection}{\bfseries\large}{\arabic{\thesection}.\arabic{\thesubsection}}{1em}{}
\titleformat{\subsubsection}{\itshape}{\arabic{\thesection}.\arabic{\thesubsection}.\arabic{\thesubsubsection}}{1em}{}

\setlength{\parindent}{0cm}
\setlength{\parskip}{0.8em}

\captionsetup[figure]{labelfont=it,font=it}
\captionsetup[table]{labelfont={it,sc},font={it,sc}}

\hypersetup{colorlinks,
    linkcolor = black,
    citecolor = black,
    urlcolor  = black}
\urlstyle{same}



\begin{document}

Soo-Hyun Yoo \\
CS472 \\
Homework 1\\
October 8, 2014

\begin{enumerate}
	\item[1.8]
		\begin{enumerate}
			\item If transistor count were to double every two years, the
				number in 2015 should be 32 times that in 2005.
			\item Starting at 30 MHz growing at 52\% every year as shown in
				Figure 1.1, clock rates in 2015 should have been around
				$30\text{ MHz} \cdot 1.52^{25} = 1\text{ THz}$.
			\item 3.3 GHz in 2010 increasing at 22\% per year should bring
				clock rates up to 8.9 GHz in 2015.
			\item As transistor size shrinks, issues such as propagation delay
				and power dissipation limit play a larger role. Rather than
				making one-time improvements to increase the clock speed,
				engineers now focus on adding more cores per chip and make
				innovations in speculative execution and caches.
			\item DRAM capacity improvements may stop by 2020.
		\end{enumerate}
	\item[1.12]
		\begin{enumerate}
			\item Assuming MTTR of 0, $35/3 = 11.67\text{ days}$
			\item Doubling the MTTF for an individual computer will double the
				MTTF of the system. For 10000 computers, the upgrade will cost
				\$10 million. Even for a multibillion-dollar company, this is
				a significant amount of money, and considering automated
				reboots of crashed servers cost very little, this upgrade makes
				no sense.
			\item $\frac{5}{4} \cdot 90000 = 112500\text{ dollars/hr}$
		\end{enumerate}
	\item[1.18]
		\begin{enumerate}
			\item Speedup will be $\cfrac{1}{0.2 + \frac{0.8}{N}}$.
			\item $\cfrac{1 - 0.005\cdot8}{0.2 + \frac{0.8}{8}} = \boxed{3.2}$
			\item $\cfrac{1 - 0.005\cdot3}{0.2 + \frac{0.8}{8}} = \boxed{3.283}$
			\item Speedup will be $\cfrac{1 - 0.005\cdot\log_2N}{0.2 + \frac{0.8}{N}}$.
			\item Speedup $s(N) = \cfrac{1 - 0.005\cdot\log_2N}{\frac{100-P}{100} + \frac{P}{100N}}$. Speedup is maximized when $s'(N) = 0$.
		\end{enumerate}
\end{enumerate}

\end{document}

