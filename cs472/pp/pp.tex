\documentclass[12pt,letterpaper]{article}
\usepackage[margin=1in]{geometry}
\usepackage{fancyhdr}
\usepackage[utf8]{inputenc}
\usepackage{palatino}
\usepackage{microtype}
\usepackage{hyperref}
\usepackage{graphicx}
\usepackage{lastpage}
\usepackage[hang,small]{caption}
\usepackage{titlesec}
\usepackage{amsmath,amssymb}
\usepackage{multirow}

\renewcommand{\headrulewidth}{0pt}
\fancyfoot{}
%\fancyfoot[C]{\sf Page \thepage\ of \pageref{LastPage}}
\pagestyle{fancy}

\titleformat{\section}{\bfseries\Large}{\arabic{\thesection}}{1em}{}
\titleformat{\subsection}{\bfseries\large}{\arabic{\thesection}.\arabic{\thesubsection}}{1em}{}
\titleformat{\subsubsection}{\itshape}{\arabic{\thesection}.\arabic{\thesubsection}.\arabic{\thesubsubsection}}{1em}{}

\setlength{\parindent}{0cm}
\setlength{\parskip}{0.8em}

\captionsetup[figure]{labelfont=it,font=it}
\captionsetup[table]{labelfont={it,sc},font={it,sc}}

\hypersetup{colorlinks,
    linkcolor = black,
    citecolor = black,
    urlcolor  = black}
\urlstyle{same}



\begin{document}

Soo-Hyun Yoo \\
CS472 \\
Project Proposal \\
October 13, 2014

% Another idea is implementing TMR in non-ECC GPUs like the GeForce series. Or
% we could just use the Tesla GPUs that already have ECC.

\section*{GPGPU hard-realtime computing}

Robotic perception and interaction involves computations that range from
complex and slow (e.g., stereo vision, pathfinding) to simple and fast (e.g.,
small feedback loops, filters). Often, the simple and fast operations occur in
the context of hard-realtime systems where jitter must be virtually reduced to
zero. Applications that require both complex behavior and realtime operation
attempt to compromise by patching the time delay of the complex behavior with
simple feedback loops. Ultimately, however, this is only a patch.

The advent of GPUs as a general-purpose computing platform has made it possible
to accelerate some of these complex algorithms. Yet, only in the last three
years has anyone put thought towards hard-realtime concerns on GPUs. Even then,
only proof-of-concept methods and benchmarks have been demonstrated.

The Xenomai realtime kernel patch will be modified to interface with CUDA or
OpenCL. Hopefully, it will be possible to bring jitter below 5 ms for tasks
offloaded to the GPU, but this will depend on the particular GPU and driver
used.

The GPU driver provides the userspace interface to run command groups on the
GPU, so the realtime scheduler must be implemented at the driver level. The
open-source Nouveau driver will be modified. One way in which to make some
semblance of a realtime guarantee may be to reserve a number of GPU cores for
the realtime threads.

It is interesting to note the Tegra K1 SoC on the Jetson includes a unified
virtual memory shared between the 4+1 CPU cores and the GPU, which resolves the
bandwidth and latency issues between the CPU and GPU in conventional systems
with discrete GPUs. This will simplify scheduling and makes the Jetson
especially suited for realtime applications.

A feasible scope for this class might be to establish the jitter of the
unmodified driver and to establish reliable task scheduling in the GPU.

Proper analysis of this problem will require performance calculations (Course
Objective 1), careful prediction of task runtimes (3, 5), use of the GPU cache
to speed realtime tasks (7), and parallel programming (9).

\end{document}

