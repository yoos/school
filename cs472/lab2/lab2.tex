\documentclass[12pt,letterpaper]{article}
\usepackage[margin=1in]{geometry}
\usepackage{fancyhdr}
\usepackage[utf8]{inputenc}
\usepackage{palatino}
\usepackage{microtype}
\usepackage{hyperref}
\usepackage{graphicx}
\usepackage{lastpage}
\usepackage[hang,small]{caption}
\usepackage{titlesec}
\usepackage{amsmath,amssymb}
\usepackage{multirow}

\renewcommand{\headrulewidth}{0pt}
\fancyfoot{}
\fancyfoot[C]{\sf Page \thepage\ of \pageref{LastPage}}
\pagestyle{fancy}

\titleformat{\section}{\bfseries\Large}{\arabic{\thesection}}{1em}{}
\titleformat{\subsection}{\bfseries\large}{\arabic{\thesection}.\arabic{\thesubsection}}{1em}{}
\titleformat{\subsubsection}{\itshape}{\arabic{\thesection}.\arabic{\thesubsection}.\arabic{\thesubsubsection}}{1em}{}

\setlength{\parindent}{0cm}
\setlength{\parskip}{0.8em}

\captionsetup[figure]{labelfont=it,font=it}
\captionsetup[table]{labelfont={it,sc},font={it,sc}}

\hypersetup{colorlinks,
    linkcolor = black,
    citecolor = black,
    urlcolor  = black}
\urlstyle{same}



\begin{document}

Soo-Hyun Yoo \\
CS472 \\
Lab 2 \\
October 24, 2014

The FLOPS per watt was determined for an 8-bit ATmega328p Arduino and a 32-bit
STM32F3 Discovery development board.

\section*{Setup}

\subsection*{8-bit Microcontroller}

An ATmega328p clocked at 16 MHz was set up with code such as the following to
toggle a GPIO pin after every floating point multiply operation.

\begin{verbatim}
void main(void) {
    DDRB |= _BV(5);

    float a = 0.0;
    float c = 99999.645543999993;
    float b = 999.99999999999999;

    while(1) {
        PORTB = ~(PORTB & _BV(5));
        a = b * c;
    }
}
\end{verbatim}

The frequency $f$ of the resulting square wave was measured with an
oscilloscope. The number of operations per second was then $2f$.

\subsection*{32-bit STM32F3 Discovery development board}

The STM32F3 devboard clocked at 72 MHz was set up similarly to the Arduino in
toggling a GPIO pin after every floating point multiply operation. The
following code was compiled with the hardware FPU module enabled.

\begin{verbatim}
int main(void) {
    halInit();
    chSysInit();

    palSetPadMode(GPIOC, 6, PAL_MODE_OUTPUT_PUSHPULL);

    volatile float a = 1.5555555555555;
    volatile float b = 1.6766763454354;
    volatile float c = 6.7865776565756;
    while(true) {
        palTogglePad(GPIOC, 6);

        a = b * c;
    }
}
\end{verbatim}

\section*{Results and Analysis}

The power consumption of each devboard was determined by multiplying the supply
voltage by the consumed current. The FLOPS/mW figure is also shown:

\begin{table}[!h]
	\centering
	\caption*{FLOPS and wattage of ATmega328p and STM32F3 devboards}
	\begin{tabular}{|l|c|c|c|} \hline
		           & FLOPS   & Power (mW) & FLOPS/mW \\ \hline\hline
		ATmega328p & 90000   & 125        & 720 \\ \hline
		STM32F3    & 5142000 & 262        & 19626 \\ \hline
	\end{tabular}
\end{table}

The Discovery board outperforms the Arduino by over an order of magnitude.

It is worth noting that the ATmega328P and STM32F3 microcontrollers themselves,
when isolated from the rest of their respective devboards, pulled 8 and 40 mA,
respectively. This translates to 11250 and 128550 FLOPS/mW, respectively, which
pushes the performance factor of the STM32F3 only slightly beyond an order of
magnitude over the ATmega328P.

\end{document}

