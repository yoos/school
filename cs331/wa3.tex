\documentclass[10pt,letterpaper]{article}
\usepackage[margin=1in]{geometry}
\usepackage{fancyhdr}
\usepackage[utf8]{inputenc}
\usepackage{palatino}
\usepackage{microtype}
\usepackage{hyperref}
\usepackage{graphicx}
\usepackage[hang,bf,small]{caption}
\usepackage{amsmath,amssymb,amsthm}

\setlength{\parindent}{0cm}
\setlength{\parskip}{1em}

\hypersetup{colorlinks,
	linkcolor = black,
	citecolor = black,
	urlcolor  = black}
\urlstyle{same}

\begin{document}

\begin{titlepage}
	\vspace*{4cm}
	\begin{flushright}
	{\huge
		Written Assignment \#3 \\ [3cm]
	}
	{\large
		CS 331
	}
	\end{flushright}

	\begin{flushright}
	Soo-Hyun Yoo
	\end{flushright}
\end{titlepage}

\section*{Written Assignment \#3}

\begin{enumerate}
	\item
		\begin{enumerate}
			\item
				\begin{tabular}{|c|c|} \hline
					{\bf Toothache} & {\bf P(Toothache)} \\ \hline
					true & 0.2 \\ \hline
					false & 0.8 \\ \hline
				\end{tabular}
			\item
				\begin{tabular}{|c|c|} \hline
					{\bf Cavity} & {\bf P(Cavity)} \\ \hline
					true & 0.2 \\ \hline
					false & 0.8 \\ \hline
				\end{tabular}
			\item
				\begin{tabular}{|c|c|} \hline
					{\bf Toothache} & {\bf P(Toothache $|$ Cavity)} \\ \hline
					true & 0.6 \\ \hline
					false & 0.4 \\ \hline
				\end{tabular}
		\end{enumerate}

	\item
		\begin{enumerate}
			\item True. We see that \[P(a,b,c) = P(a|b,c)P(b|c)P(c)
				= P(b|a,c)P(a|c)P(c).\] Since we are given \[P(a|b,c)
				= P(b|a,c),\] \[P(a|c) = P(b|c),\] as desired.
			\item False. Since $P(a|b,c)=P(a)$, $P(a)$ is independent of $b$
				and $c$, but this does not tell us anything about the
				relationship between $b$ and $c$. As a counterexample consider
				the case when the antecedent is true and $P(b|c) \neq P(b)$.
				Then (obviously) the consequent is false.
			\item True. Since $P(a)$ is independent of $b$, we can simply
				disregard it in $P(a|b,c)$, so $P(a|b,c)=P(a|c)$.
		\end{enumerate}

	\item The full joint probability distribution for the two tests are as follows:

		\begin{table}[!h]
			\centering
			\begin{tabular}{|c|c|c|} \hline
				{\bf V} & {\bf A} & {\bf P(A,V)} \\ \hline\hline
				T & T & 0.0095 \\ \hline
				T & F & 0.0005 \\ \hline
				F & T & 0.099  \\ \hline
				F & F & 0.891  \\ \hline
			\end{tabular}
			\hspace{1cm}
			\begin{tabular}{|c|c|c|} \hline
				{\bf V} & {\bf B} & {\bf P(B,V)} \\ \hline\hline
				T & T & 0.009  \\ \hline
				T & F & 0.001  \\ \hline
				F & T & 0.0495 \\ \hline
				F & F & 0.9405 \\ \hline
			\end{tabular}
		\end{table}

		If Test A comes back positive, the probability that the person is
		really carrying the virus is $\frac{0.0095}{0.0095+0.099} = 0.0876$. On
		the other hand, if Test B comes back positive, that probability is
		$\frac{0.009}{0.009+0.0495} = 0.1538$. Thus, $\boxed{\text{Test B}}$ is
		more indicative of a person really carrying the virus.

	\item
		\begin{enumerate}
			\item If the coin is fake, the probability that it flips to head is
				$1$. Otherwise, the probability is $\frac12$. Since there is
				only $1$ fake coin among $n$ coins, the probability that the
				coin I picked is fake is
				$\frac{\frac{1}{n}}{\frac{n-1}{n}\cdot\frac12}
				= \boxed{\frac{2n}{n(n-1)}}$.
			\item The probability that a real coin flips to head $k$ times out
				of $k$ is $\frac{1}{2^k}$. Thus, after $k$ flips of $k$ heads,
				the probability that the coin is fake is
				$\frac{\frac{1}{n}}{\frac{n-1}{n}\cdot\frac{1}{2^k}}
				= \boxed{\frac{2^kn}{n(n-1)}}$.
		\end{enumerate}
\end{enumerate}

\end{document}

