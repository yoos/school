\documentclass[10pt,letterpaper]{article}
\usepackage[margin=1in]{geometry}
\usepackage{fancyhdr}
\usepackage[utf8]{inputenc}
\usepackage{palatino}
\usepackage{microtype}
\usepackage{hyperref}
\usepackage{graphicx}
\usepackage[hang,bf,small]{caption}
\usepackage{amsmath,amssymb,amsthm}

\setlength{\parindent}{0cm}
\setlength{\parskip}{1em}

\hypersetup{colorlinks,
	linkcolor = black,
	citecolor = black,
	urlcolor  = black}
\urlstyle{same}

\begin{document}

\begin{titlepage}
	\vspace*{4cm}
	\begin{flushright}
	{\huge
		Written Assignment \#2 \\ [3cm]
	}
	{\large
		CS 331
	}
	\end{flushright}

	\begin{flushright}
	Soo-Hyun Yoo
	\end{flushright}
\end{titlepage}

\section*{Written Assignment \#2}

\begin{enumerate}
	\item
		\begin{enumerate}
			\item True. The antecedent is false, so we can make up
				anything we want in the consequent.
			\item True. The consequent is equivalent to $(A \land B) \lor (\neg A \land \neg B)$.
			\item True. This can be satisfied with $A$ and $\neg B$.
		\end{enumerate}

	\item
		\begin{enumerate}
			\item No.
			\item Yes. The unicorn is either immortal or a mortal mammal, so it must be horned, which makes it magical.
			\item Yes, as explained in (b).
		\end{enumerate}

	\item
		\begin{enumerate}
			\item Valid. If Smoke is given, Smoke must be true.
			\item Invalid. The inverse is false.
			\item Valid. Smoke is still given in te consequent, so Fire follows.
		\end{enumerate}

	\item
		\begin{enumerate}
			\item
				\begin{align*}
					[(Food \Rightarrow Party) \lor (Drinks \Rightarrow Party)] &\Rightarrow [(Food \land Drinks) \Rightarrow Party] \\
					[(\neg Food \lor Party) \lor (\neg Drinks \lor Party)] &\Rightarrow [\neg (Food \land Drinks) \lor Party] \\
					(\neg Food \lor \neg Drinks \lor Party) &\Rightarrow (\neg Food \lor \neg Drinks \lor Party)
				\end{align*}
			\item The two sides are equivalent, which makes the statement valid.
		\end{enumerate}
\end{enumerate}

\end{document}

