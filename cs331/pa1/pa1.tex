\documentclass[12pt,letterpaper]{article}
\usepackage[margin=1in]{geometry}
\usepackage{fancyhdr}
\usepackage[utf8]{inputenc}
\usepackage{palatino}
\usepackage{microtype}
\usepackage{hyperref}
\usepackage{graphicx}
\usepackage[hang,bf,small]{caption}
\usepackage{amsmath,amssymb,amsthm}

\setlength{\parindent}{0cm}
\setlength{\parskip}{1em}

\hypersetup{colorlinks,
	linkcolor = black,
	citecolor = black,
	urlcolor  = black}
\urlstyle{same}

\begin{document}

\begin{titlepage}
	\vspace*{4cm}
	\begin{flushright}
	{\huge
		Programming Assignment \#1 \\ [3cm]
	}
	{\large
		CS 331
	}
	\end{flushright}

	\begin{flushright}
		19 April 2013
		Soo-Hyun Yoo
	\end{flushright}
\end{titlepage}

\section*{Programming Assignment \#1}

\subsection*{Methodology}

% Methodology: Describe the experiments you ran on the three test cases. Make
% sure you specify all parameters that you used for the search algorithms (eg.
% the depth limit in DFS, heuristic for A-star search). You can assume the
% reader is familiar with search algorithms and you don't need to describe how
% the search algorithms operate. However, the description should be complete
% enough that someone else can reimplement these search algorithms on their own
% and re-run the exact same set of experiments using the parameters you used.
% Make sure you explain why you chose the heuristic for A-star search.

There was no depth limit to either the DFS or IDDFS algorithms.

For $n$ cannibals and $m$ missionaries, the heuristic used for the A* search
was to minimize $2(n+m)-1$ on the right bank. If we remove the constraint that
cannibals must not outnumber a positive number of missionaries, we can assign
one person to row the boat back and forth and transport one person at a time.
The boat must make $2(n+m-1)$ trips for everyone but the person rowing the
boat, and the person rowing the boat needs only one trip, for a total minimum
of $2(n+m-1)+1 = 2(n+m)-1$ trips.


\subsection*{Results}

% Results: Summarize your results for the three test cases in terms of the # of
% nodes on the solution path and # of nodes expanded on each test graph for
% each of the search techniques. Use either a table or a graph to illustrate
% the results.

The UNIX {\tt time} utility was used to time the algorithms. {\tt time}'s
accuracy of 1 ms was deemed sufficient for the timescales involved.

The solution depth, number of nodes expanded, and the time taken by each
algorithm for the three cases are summarized in the tables below.

\begin{table}[!h]
	\centering
	\begin{tabular}{|l|r|r|r|} \hline
		Search Algorithm & Solution Depth & Nodes Expanded & Time (s) \\ \hline
		BFS   & 11 & 15 & 0.11 \\
		DFS   & 11 & 12 & 0.11 \\
		IDDFS & 11 & 98 & 0.13 \\
		A*    & 11 & 13 & 0.11 \\ \hline
	\end{tabular}
	\caption{Results of search algorithms on test case 1.}
	\label{tbl:case1}
\end{table}

\begin{table}[!h]
	\centering
	\begin{tabular}{|l|r|r|r|} \hline
		Search Algorithm & Solution Depth & Nodes Expanded & Time (s) \\ \hline
		BFS   & 33 & 73 & 0.12 \\
		DFS   & 33 & 47 & 0.12 \\
		IDDFS & 33 & 1391 & 0.35 \\
		A*    & 33 & 47 & 0.12 \\ \hline
	\end{tabular}
	\caption{Results of search algorithms on test case 2.}
	\label{tbl:case2}
\end{table}

\begin{table}[!h]
	\centering
	\begin{tabular}{|l|r|r|r|} \hline
		Search Algorithm & Solution Depth & Nodes Expanded & Time (s) \\ \hline
		BFS   & 377 & 2185 & 0.54 \\
		DFS   & 473 & 1309 & 0.37 \\
		IDDFS & 471 & 799719 & 112.59 \\
		A*    & 377 & 709 & 0.38 \\ \hline
	\end{tabular}
	\caption{Results of search algorithms on test case 3.}
	\label{tbl:case3}
\end{table}


\subsection*{Discussion}

% Discussion: Discuss your results. Are they as expected? Was there any
% interesting behavior that you found?

All four algorithms found the optimal solution for Case 1 and Case 2.

BFS and A* always found the optimal solution, as expected.

DFS and A* expanded similar numbers of nodes for the smaller two cases, but A*
was the most space-efficient for Case 3.

IDDFS found a slightly more optimal solution to Case 3 than did DFS, probably
due to its BFS-like search pattern.


\subsection*{Conclusion}

% Conclusion: Add a conclusion which answers the following questions: What can
% you conclude from these results? Which search algorithm performs the best?
% Was this expected?

For small test cases, the type of algorithm used does not significantly affect
time and space requirements.

The A* search algorithm performed the best, based on the fact that it always
found the optimal solution with the least number of nodes expanded (almost,
excluding Case 1 where DFS beat it by 1 node). This was expected but still
surprising and awesome!


\end{document}

