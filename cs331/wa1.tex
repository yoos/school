\documentclass[10pt,letterpaper]{article}
\usepackage[margin=1in]{geometry}
\usepackage{fancyhdr}
\usepackage[utf8]{inputenc}
\usepackage{palatino}
\usepackage{microtype}
\usepackage{hyperref}
\usepackage{graphicx}
\usepackage[hang,bf,small]{caption}
\usepackage{amsmath,amssymb,amsthm}

\setlength{\parindent}{0cm}
\setlength{\parskip}{1em}

\hypersetup{colorlinks,
	linkcolor = black,
	citecolor = black,
	urlcolor  = black}
\urlstyle{same}

\begin{document}

\begin{titlepage}
	\vspace*{4cm}
	\begin{flushright}
	{\huge
		Written Assignment \#1 \\ [3cm]
	}
	{\large
		CS 331
	}
	\end{flushright}

	\begin{flushright}
	Soo-Hyun Yoo
	\end{flushright}
\end{titlepage}

\section*{Written Assignment \#1}

\begin{enumerate}
	\item
		\begin{enumerate}
			\item
				\begin{itemize}
					\item Performance Measure: Cheapest, usable condition, fast
						shipping, good customer service
					\item Environment: Vendors, other buyers
					\item Actuators: Capability of crawling the web
					\item Sensors: Capability of parsing web pages
				\end{itemize}
			\item
				\begin{itemize}
					\item Partially observable (cannot see all of the Internet
						at once)
					\item Strategic (other buyers are making purchases at
						unknown times)
					\item Sequential (a decision to reconsider a listing later
						may result in the item being sold to another buyer)
					\item Dynamic (the Internet is always changing)
					\item Continuous (purchases can be made at any time)
					\item Single agent (although there are other buyers, we are
						dealing with only a single agent we can control)
				\end{itemize}
			\item A utility-based agent would be most appropriate here.
				A simple reflex agent will be unable to compare all available
				deals. A model-based reflex agent may be able to find available
				deals but be unable to pick out the best. A goal-based agent is
				not useful in determining the best deal (it may be useful in
				finding the deals, but that is not the focus here).
				A utility-based agent will be able to weigh the different
				performance measures and decide on the best deal, which may or
				may not maximize any one of the performance measures but result
				in an overall ``happiest'' state.
		\end{enumerate}

	\item
		\begin{enumerate}
			\item Yes. A lookup table is one way with which to map a percept
				sequence to action. On the other hand, a function may be able
				to do the same without the need for a table stored in memory.
			\item Yes. No agent program can implement the agent function of
				breaking out of an enclosed maze, or calculating the last digit
				of $\sqrt{2}$.
			\item Yes. If the agent program does not change, its function
				cannot change.
			\item $2^n$ agent programs.
			\item It depends. If the agent program implements a wait function
				that depends on the number of processor cycles as its timer,
				the agent function will change. On the other hand, if the agent
				function sat and did nothing, no change to the machine would
				change the agent function.
		\end{enumerate}

	\item
		\begin{enumerate}
			\item Yes. An agent function that cleans the square if it is dirty
				and moves on otherwise could allow the robot to eventually
				cover the entire environment, regardless of extent, boundaries,
				or obstacles.
			\item Yes. A deterministic agent function may become stuck in
				a loop and consequently miss parts of the environment.
				A randomized agent function has at least a chance of covering
				the entire environment.
			\item Yes. An agent with state can distinguish between two global
				states with the same percepts but which are actually different
				situations. This can further help the agent avoid infinite
				loops.
		\end{enumerate}
\end{enumerate}

\end{document}

