\documentclass[12pt,letterpaper]{article}
\usepackage[margin=1in]{geometry}
\usepackage{fancyhdr}
\usepackage[utf8]{inputenc}
\usepackage{palatino}
\usepackage{microtype}
\usepackage{hyperref}
\usepackage{graphicx}
\usepackage{lastpage}
\usepackage[hang,small,margin=1in]{caption}
\usepackage{titlesec}
\usepackage{pdfpages}

\renewcommand{\headrulewidth}{0pt}
\fancyfoot{}
\fancyfoot[C]{\sffamily Page \thepage\ of \pageref{LastPage}}
\pagestyle{fancy}

\titleformat{\section}{\bfseries\MakeUppercase}{\arabic{\thesection}}{1em}{}
\titleformat{\subsection}{\bfseries}{\arabic{\thesection}.\arabic{\thesubsection}}{1em}{}
\titleformat{\subsubsection}{\itshape}{\arabic{\thesection}.\arabic{\thesubsection}.\arabic{\thesubsubsection}}{1em}{}

\setlength{\parindent}{0cm}
\setlength{\parskip}{1em}

\captionsetup[figure]{labelfont=it, font=it}
\captionsetup[table]{labelfont={it,sc}, font={it,sc}}

\hypersetup{colorlinks, linkcolor = black, citecolor = black, urlcolor = black}
\urlstyle{same}



\begin{document}

\fancyfoot{}
\begin{center}
    \hfill \\
    \vspace{4in}
    {\bf\Huge CS480 Milestone \#3 \\}
    \vspace{2in}
    {\Large Soo-Hyun Yoo \\ February 16, 2015}
\end{center}

\newpage
\fancyfoot[C]{\sffamily Page \thepage\ of \pageref{LastPage}}

% Handwritten answers
%\includepdf[pages={1}]{m3_handwritten.pdf}

\section*{Specification}

This milestone prompts me to pick a data structure to represent the abstract
syntax graphs that will be used to determine whether or not the list of symbols
satisfy the syntax of the IBTL. In doing so, I will also find out whether the
symbol table structure is sufficient for the tasks at hand or needs
improvement.

\section*{Processing}

All token types were recategorized as lists of types and the type symbols used
to store the (type, string) tuples in the symbol table.

The provided grammar was refactored to more closely mirror the lexer before
being incorporated into the grammar definition in code. The compiler passes
this definition to the syntax parser along with the symbol table output from
the lexer and the initial recursion depth of 0.

\section*{Testing Requirement}

% TODO

I tested the lexer for correctness using the same inputs used in Milestone 2 to
test the lexer. The inputs are composed of as many combinations of token types
as I could reasonably come up with. The tokens are output at an indentation
proportional to the recursion depth, which was checked manually.

\section*{Retrospective}

The symbol table and lexer did not require heavy modification. I can see
(again) that having a clear model and modular code is paramount to keeping the
compiler comprehensible.

\end{document}
