\documentclass[12pt,letterpaper]{article}
\usepackage[margin=1in]{geometry}
\usepackage{fancyhdr}
\usepackage[utf8]{inputenc}
\usepackage{palatino}
\usepackage{microtype}
\usepackage{hyperref}
\usepackage{graphicx}
\usepackage{lastpage}
\usepackage[hang,small,margin=1in]{caption}
\usepackage{titlesec}
\usepackage{pdfpages}

\renewcommand{\headrulewidth}{0pt}
\fancyfoot{}
\fancyfoot[C]{\sffamily Page \thepage\ of \pageref{LastPage}}
\pagestyle{fancy}

\titleformat{\section}{\bfseries\MakeUppercase}{\arabic{\thesection}}{1em}{}
\titleformat{\subsection}{\bfseries}{\arabic{\thesection}.\arabic{\thesubsection}}{1em}{}
\titleformat{\subsubsection}{\itshape}{\arabic{\thesection}.\arabic{\thesubsection}.\arabic{\thesubsubsection}}{1em}{}

\setlength{\parindent}{0cm}
\setlength{\parskip}{1em}

\captionsetup[figure]{labelfont=it, font=it}
\captionsetup[table]{labelfont={it,sc}, font={it,sc}}

\hypersetup{colorlinks, linkcolor = black, citecolor = black, urlcolor = black}
\urlstyle{same}



\begin{document}

\fancyfoot{}
\begin{center}
    \hfill \\
    \vspace{4in}
    {\bf\Huge CS480 Milestone \#4 \\}
    \vspace{2in}
    {\Large Soo-Hyun Yoo \\ March 2, 2015}
\end{center}

\newpage
\fancyfoot[C]{\sffamily Page \thepage\ of \pageref{LastPage}}

\section*{Formal Definition}

\subsection*{Semantics Parser (Constants Only)}

We can assume the input is well-formed, as the syntax parser will have thrown
an error otherwise. The following defines a recursive semantics parser:

\begin{verbatim}
def sem-parse(parse-tree):
    if parse-tree.type is token:
        return parse-tree
    else:
        op = parse-tree.pop()
        if op is unop:
            type, code = gen-code(op,
                                  sem-parse(parse-tree.pop()))
        else:
            type, code = gen-code(op,
                                  sem-parse(parse-tree.pop()),
                                  sem-parse(parse-tree.pop()))
        return make-token(type, code)
\end{verbatim}

\section*{Specification}

This milestone gives me practice producing gforth code from IBTL by
implementing a semantics parser for constants only. A simple semantics parser
will be written to recursively parse the parse tree and generate gforth code.
This will hopefully shed some light on how to properly build the parser for the
other token types in the next milestone.

\section*{Processing}

The semantics parser was updated to fully reject any malformed input. That is,
all operators were verified for the correct number of associated operands and
that constants were not being used as operators, etc. The parser could then
generate the gforth code by recursively adding onto several stacks representing
the various gforth stacks.

\section*{Testing Requirement}

I made short, simple test cases to test all operators for all four constant
types. The simplicity of each test case allows for manual verification of test
results. The test cases are grouped by their expected result (success/failure).

\section*{Retrospective}

The resultant semantics parser did not require a crazy network of state
machines as I had previously envisioned (partially due to an initial
misunderstanding of how the work is divided between the syntax and semantics
parsers). An attempt was made at making an LR(1) parser, but I quickly realized
a parser generator would be much better suited to such a task.

\end{document}
