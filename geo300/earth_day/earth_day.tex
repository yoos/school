\documentclass[12pt,letterpaper]{article}
\usepackage[margin=1in]{geometry}
\usepackage{fancyhdr}
\usepackage[utf8]{inputenc}
\usepackage{palatino}
\usepackage{microtype}
\usepackage{hyperref}
\usepackage{graphicx}
\usepackage{lastpage}
\usepackage[hang,small,margin=1in]{caption}
\usepackage{titlesec}
\usepackage[authoryear,comma,sort&compress]{natbib}

\renewcommand{\headrulewidth}{0pt}
\fancyfoot{}
\fancyfoot[C]{\sffamily Page \thepage\ of \pageref{LastPage}}
\pagestyle{fancy}

\titleformat{\section}{\bfseries\MakeUppercase}{\arabic{\thesection}}{1em}{}
\titleformat{\subsection}{\bfseries}{\arabic{\thesection}.\arabic{\thesubsection}}{1em}{}
\titleformat{\subsubsection}{\itshape}{\arabic{\thesection}.\arabic{\thesubsection}.\arabic{\thesubsubsection}}{1em}{}

\setlength{\parindent}{0cm}
\setlength{\parskip}{1em}

\captionsetup[figure]{labelfont=it, font=it}
\captionsetup[table]{labelfont={it,sc}, font={it,sc}}

\hypersetup{colorlinks, linkcolor = black, citecolor = black, urlcolor = black}
\urlstyle{same}

\renewcommand{\citep}[1]{(\cite{#1})}


\begin{document}

%\fancyfoot{}
%\begin{center}
%    \hfill \\
%    \vspace{4in}
%    {\bf\Huge GEO300 CT2 \\}
%    \vspace{2in}
%    {\Large Soo-Hyun Yoo \\ April 24, 2015}
%\end{center}
%
%\newpage
%\fancyfoot[C]{\sffamily Page \thepage\ of \pageref{LastPage}}

\section*{Earth Day}

Many consider Earth Day 1970, which has since occurred every April 22, to have
birthed the modern environmental movement. Years of wartime industrial growth
had caused extensive damage to U.S. waterways and natural resources, especially
given a lack of regulation for such activities. Factories liberally spilled
toxic wastes into rivers. Synthetic chemicals such as DDT were used with little
concern for their environmental and health consequences~\citep{green2015}.

Gaylord Nelson, a Wisconsin senator, founded Earth Day by channeling the
American population's anti-war energy towards this increasingly evident
environmental degradation. In an age where the smell of pollution was
considered the smell of prosperity, Nelson managed to lead the nation in
a united effort to preserve the environment~\citep{edn2015}. Nelson's founding
of Earth Day conveniently followed several environmental catastrophes: in
November 1966, 168 people died in New York City due to horrible air quality; in
1969, an oil rig off the shore of Santa Barbara exploded, spilling 10,000
gallons of crude oil into the surrounding water and beaches; also in 1969, the
Cuyahoga river running through Cleveland burst into flames due to contaminants
in the water~\citep{green2015}.

Student response to these catastrophes was fierce, and they were particularly
involved in spreading the idea of Earth Day and the awareness of environmental
degradation. Some of the first Earth Day demonstrations involved many mock
trials and funerals of polluting objects. At Florida Technological University,
for example, students charged a Chevrolet and found it guilty of poisoning the
air and proceeded to execute it with a sledgehammer. By year's end, Congress
had authorized the creation of the Environmental Protection Agency, and by the
following year, the event even gained the President's official
sanction~\citep{latson2015}.

The newly formed Environmental Protection Agency reported that since 1969, when
polled in 1971, 25 times as many people (for a total of 25 percent of the U.S.
population) declared environmental protection to be an important
goal~\citep{history2009}.

In a journal in 1980, Nelson stated that his ``primary objective in planning
Earth Day was to show the political leadership of the Nation that there was
broad and deep support for the environmental movement.'' Within the decade
following the first Earth Day in 1970, numerous laws were passed to protect the
environment~\citep{nelson1980} following public opinion newly brought to front
stage.

The movement has since continued to grow. In 1990, more than 200 million people
from over 140 nations participated in Earth Day
celebrations~\citep{history2009}.  \cite{edn2015}, which coordinates Earth Day
celebrations around the world, claims that more than a billion people are now
involved in Earth Day activities.

\bibliographystyle{apa}
\bibliography{earth_day.bib}

\end{document}
