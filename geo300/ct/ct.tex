\documentclass[12pt,letterpaper]{article}
\usepackage[margin=1in]{geometry}
\usepackage{fancyhdr}
\usepackage[utf8]{inputenc}
\usepackage{palatino}
\usepackage{microtype}
\usepackage{hyperref}
\usepackage{graphicx}
\usepackage{lastpage}
\usepackage[hang,small,margin=1in]{caption}
\usepackage{titlesec}
\usepackage[authoryear,colon,sort&compress]{natbib}

\renewcommand{\headrulewidth}{0pt}
\fancyfoot{}
\fancyfoot[C]{\sffamily Page \thepage\ of \pageref{LastPage}}
\pagestyle{fancy}

\titleformat{\section}{\bfseries\MakeUppercase}{\arabic{\thesection}}{1em}{}
\titleformat{\subsection}{\bfseries}{\arabic{\thesection}.\arabic{\thesubsection}}{1em}{}
\titleformat{\subsubsection}{\itshape}{\arabic{\thesection}.\arabic{\thesubsection}.\arabic{\thesubsubsection}}{1em}{}

\setlength{\parindent}{0cm}
\setlength{\parskip}{1em}

\captionsetup[figure]{labelfont=it, font=it}
\captionsetup[table]{labelfont={it,sc}, font={it,sc}}

\hypersetup{colorlinks, linkcolor = black, citecolor = black, urlcolor = black}
\urlstyle{same}


\begin{document}

Soo-Hyun Yoo \\
930-569-466 \\
GEO 300, F12 \\
Sandra Huynh \\
CT3 due 5/8/15 \\
Word Count: 435

\section*{Hybrid and Electric Cars Will Not Solve World's Oil Addiction}

\subsection*{Interpretation (19 words)}   % 25 words max

Current oil consumption patterns indicate the use of hybrid and electric cars
will not solve the world's oil addiction.

\subsection*{Analysis (317 words)}   % 300-350 words

Privately owned vehicles are mainly to blame for the world's heavy dependence
on oil. However, the small efficiency gains auto manufacturers may be able to
eke out of hybrid vehicles are not enough to counteract heavy American
consumerism and the rapidly increasing demand for oil in developing countries
whose populace wish to emulate the Western way of life~\citep[p.
310]{yetiv2011}. In fact, \citet[p. 289]{yetiv2011} found that even if the U.S.
were to somehow convert to 100 percent hybrid car use in 2011, the Chinese oil
demand would exceed whatever gain achieved through hybrid car use.

Unless the world completely revamps its method of energy production, it will
never be able to resolve its dependence on oil. Solving the oil addiction
crisis is not only tough, but a sensitive topic due to the close tie between
economic growth and energy consumption~\citep[p. 1355]{halkos2011}. Not even
the development of completely electric vehicles is enough, as the electricity
used to power such vehicles currently are generated by coal or gas-powered
plants.

Growing countries such as Malaysia shape their energy structure and
industrialization process around increasing oil consumption. Between 1980 and
2009, China, South Korea, and the Middle East and North African countries have
demonstrated a bidirectional relationship between oil consumption and economic
growth~\citep[p. 220]{park2014}. However, \citet[p. 220]{park2014} also point
out that this is not a universal fact and that in New Zealand, Shanghai, and
India, no such causality was observed. Finally, Taiwan and Pakistan have
undergone economic growth without a necessary increase in oil consumption,
indicating that economic growth is possible without the need for more
oil~\citep[p. 222]{park2014}.

With the right incentives, the U.S. could lead the world in solving the oil
crisis by example. The first step will be to bring widespread awareness, at
least at the national level, of the finiteness of the world's oil
supply~\citep[p. 289]{yetiv2011}. This need not mean a complete cessation of
oil usage -- rather, the public should understand the progress of different
developing economies around the world and help others become efficient,
responsible users of the earth's resources~\citep[p. 1361]{halkos2011}.

\subsection*{Evaluation (49 words)}   % 50 words max

Yetiv's 2011 study made a prediction for the following half decade.  Given the
accelerating economic growth rate of many countries, the study may be outdated
today. Halkos is likely unbiased due to his study being comparative. Park and
Yoo's 2014 evaluation is on another country (Malaysia), so likely unbiased.

\subsection*{Inference (50 words)}   % 50 words max

A global plan for decreasing oil demand could help the world as a whole consume
less oil. Locally, however, there will be exceptions, as some communities may
have less resources for revamping infrastructure than others.  Prominent
communities will have to execute policy changes first to set an example for
others.

\subsection*{Sources to be used in CT4}

\begin{itemize}
  \item \citet{gately2012}
\end{itemize}

\bibliographystyle{chicago}
\bibliography{ct.bib}

\end{document}
