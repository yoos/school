\documentclass[12pt,letterpaper]{article}
\usepackage[margin=1in]{geometry}
\usepackage{fancyhdr}
\usepackage[utf8]{inputenc}
\usepackage{palatino}
\usepackage{microtype}
\usepackage{hyperref}
\usepackage{graphicx}
\usepackage{lastpage}
\usepackage[hang,small,margin=1in]{caption}
\usepackage{titlesec}
\usepackage[authoryear,comma,sort&compress]{natbib}

\renewcommand{\headrulewidth}{0pt}
\fancyfoot{}
\fancyfoot[C]{\sffamily Page \thepage\ of \pageref{LastPage}}
\pagestyle{fancy}

\titleformat{\section}{\bfseries\MakeUppercase}{\arabic{\thesection}}{1em}{}
\titleformat{\subsection}{\bfseries}{\arabic{\thesection}.\arabic{\thesubsection}}{1em}{}
\titleformat{\subsubsection}{\itshape}{\arabic{\thesection}.\arabic{\thesubsection}.\arabic{\thesubsubsection}}{1em}{}

\setlength{\parindent}{0cm}
\setlength{\parskip}{1em}

\captionsetup[figure]{labelfont=it, font=it}
\captionsetup[table]{labelfont={it,sc}, font={it,sc}}

\hypersetup{colorlinks, linkcolor = black, citecolor = black, urlcolor = black}
\urlstyle{same}

\renewcommand{\citep}[1]{(\cite{#1})}


\begin{document}

%\fancyfoot{}
%\begin{center}
%    \hfill \\
%    \vspace{4in}
%    {\bf\Huge GEO300 CT2 \\}
%    \vspace{2in}
%    {\Large Soo-Hyun Yoo \\ April 24, 2015}
%\end{center}
%
%\newpage
%\fancyfoot[C]{\sffamily Page \thepage\ of \pageref{LastPage}}

Soo-Hyun Yoo \\
930-569-466 \\
GEO 300, F12 \\
Sandra Huynh \\
CT2 due 4/24/15 \\
Word Count: 311

\section*{Hybrid and Electric Cars Will Not Solve Our Oil Addiction}

\subsection*{Interpretation (18 words)}

Current oil consumption patterns indicate the use of hybrid and electric cars
will not solve our oil addiction.

\subsection*{Analysis (245 words)}

Privately owned vehicles are mainly to blame for our heavy dependence on oil.
Despite what the small efficiency gains auto manufacturers may be able to eke
out of hybrid vehicles, such gains are not enough to counteract heavy American
consumerism and the rapidly increasing demands for oil in developing countries
whose populace wish to emulate the Western way of life \citep{yetiv2011}. In
fact, \cite{yetiv2011} found that even if the U.S. were to somehow convert to
100 percent hybrid car use in 2011, the Chinese oil demand would exceed
whatever gain achieved through hybrid car use.

Unless we completely revamp our method of energy production, we will never be
able to resolve our dependence on oil. Solving the oil addiction crisis is not
only tough, but a sensitive topic due to the close tie between economic growth
and energy consumption~\citep{halkos2011}. Not even the development of
completely electric vehicles is enough, as the electricity used to power such
vehicles currently are generated by coal or gas-powered plants.

With the right incentives, the U.S. could lead the world in solving the oil
crisis by example. The first step will be to bring widespread awareness, at
least at the national level, of the finiteness of the world's oil
supply~\citep{yetiv2011}. This need not mean that we stop using oil altogether
as an energy source -- rather, we should understand the progress of different
developing economies around the world and help each other become efficient,
responsible use of the earth's resources~\citep{halkos2011}.

\subsection*{Evaluation (48 words)}

Yetiv and Fowler's study was done in 2011 and made a prediction for the
following half decade. Given the accelerating rate of many developing
countries' economic growth, the study may be outdated in 2015. Halkos and
Tzeremes are likely unbiased due to their study being a comparative one.

\subsection*{Sources to be used in CT3}

\begin{itemize}
  \item \cite{park2014}
  \item \cite{gately2012}
\end{itemize}

\bibliographystyle{apa}
\bibliography{ct.bib}

\end{document}
