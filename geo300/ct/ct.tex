\documentclass[12pt,letterpaper]{article}
\usepackage[margin=1in]{geometry}
\usepackage{fancyhdr}
\usepackage[utf8]{inputenc}
\usepackage{palatino}
\usepackage{microtype}
\usepackage{hyperref}
\usepackage{graphicx}
\usepackage{lastpage}
\usepackage[hang,small,margin=1in]{caption}
\usepackage{titlesec}
\usepackage[authoryear,colon,sort&compress]{natbib}

\renewcommand{\headrulewidth}{0pt}
\fancyfoot{}
\fancyfoot[C]{\sffamily Page \thepage\ of \pageref{LastPage}}
\pagestyle{fancy}

\titleformat{\section}{\bfseries\MakeUppercase}{\arabic{\thesection}}{1em}{}
\titleformat{\subsection}{\bfseries}{\arabic{\thesection}.\arabic{\thesubsection}}{1em}{}
\titleformat{\subsubsection}{\itshape}{\arabic{\thesection}.\arabic{\thesubsection}.\arabic{\thesubsubsection}}{1em}{}

\setlength{\parindent}{0cm}
\setlength{\parskip}{1em}

\captionsetup[figure]{labelfont=it, font=it}
\captionsetup[table]{labelfont={it,sc}, font={it,sc}}

\hypersetup{colorlinks, linkcolor = black, citecolor = black, urlcolor = black}
\urlstyle{same}


\begin{document}

Soo-Hyun Yoo \\
930-569-466 \\
GEO 300, F12 \\
Sandra Huynh \\
CT4 due 5/22/15 \\
Word Count: 550

\section*{Electric Cars Alone Will Not Solve World's Oil Addiction}

\subsection*{Interpretation (15 words)}   % 25 words max

Current oil consumption patterns indicate electric cars alone will not solve
the world's oil addiction.

\subsection*{Analysis (407 words)}   % 400-450 words

Privately owned vehicles are mainly to blame for the world's heavy dependence
on oil. However, the small efficiency gains auto manufacturers eke out of
hybrid vehicles are not enough to counteract heavy American consumerism and the
rapidly increasing demand for oil in developing countries wishing to emulate
the Western way of life~\citep[p. 310]{yetiv2011}. In fact, \citet[p.
289]{yetiv2011} found that even if the U.S. were to somehow convert to 100
percent hybrid car use in 2011, Chinese oil demand alone would exceed whatever
gains are achieved through hybrid car use.

Unless the world completely revamps its energy production methods, it will
never be able to resolve its dependence on oil. Solving the oil addiction
crisis is a tough and sensitive topic due to the close tie between economic
growth and energy consumption~\citep[p. 1355]{halkos2011}. Even electric
vehicles have a carbon footprint in the gas-powered plants that generate the
electricity.

Growing countries such as Malaysia shape their energy structure and
industrialization process around increasing oil consumption. Between 1980 and
2009, economic growth in China, South Korea, and the Middle East and North
African countries have demonstrated a bidirectional relationship between oil
consumption and economic growth~\citep[p. 220]{park2014}. Economic growth in
Saudi Arabia, the second largest producer of oil in the world, is so fast that
demand for oil within the country is outpacing increases in the rate of oil
production~\citep[p. 60]{gately2012}. Considering only a quarter of the
country's population currently owns a vehicle (compared to over 80 percent in
the U.S.), the inevitable rapid increase in domestic oil consumption in Saudi
Arabia raises concern over its ability to supply the rest of the world's
increasing demand for oil, let alone maintain current export levels~\citep[p.
65]{gately2012}.

However, \citet[p. 220]{park2014} point out that this interdependence between
oil consumption and economic growth is not a universal fact and that in New
Zealand, Shanghai, and India, no such causality was observed. Furthermore,
Taiwan and Pakistan have undergone economic growth without a necessary increase
in oil consumption, indicating that economic growth is possible without the
need for more oil~\citep[p. 222]{park2014}.

With the right incentives, the U.S. could lead the world in solving the oil
crisis by example. The first step will be to bring widespread awareness, at
least at the national level, of the finiteness of the world's oil
supply~\citep[p. 289]{yetiv2011}. This need not mean a complete cessation of
oil usage -- rather, the public should understand the progress of different
developing economies around the world and help others become efficient,
responsible users of the earth's resources~\citep[p. 1361]{halkos2011}.

\subsection*{Evaluation (41 words)}   % 50 words max

Yetiv's 2011 prediction applied to the following half decade, which may be
outdated today. The other authors are likely unbiased due to their lack of
affiliation with the subjects of their studies, which makes them a good source
of objective information.

\subsection*{Inference (43 words)}   % 50 words max

A global plan for decreasing oil demand is needed for coordination. Locally,
there will be exceptions, as some communities may have less resources for
revamping infrastructure than others. Prominent communities will have to
execute policy changes first to set an example for others.

\subsection*{Self-Regulation (44 words)}   % 50 words max

As someone whose livelihood is not built around oil, I can easily argue for
eradicating its usage. Having enjoyed the oil-powered benefits of the developed
world my whole life, I am disconnected from those who are just beginning to
share in the same convenience.

\newpage
\bibliographystyle{chicago}
\bibliography{ct.bib}

\end{document}
