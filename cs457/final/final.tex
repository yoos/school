\documentclass[12pt,letterpaper]{article}
\usepackage[margin=1in]{geometry}
\usepackage{fancyhdr}
\usepackage[utf8]{inputenc}
\usepackage{palatino}
\usepackage{microtype}
\usepackage{hyperref}
\usepackage{graphicx}
\usepackage{lastpage}
\usepackage[hang,small,margin=1in]{caption}
\usepackage{titlesec}

\renewcommand{\headrulewidth}{0pt}
\fancyfoot{}
\fancyfoot[C]{\sffamily Page \thepage\ of \pageref{LastPage}}
\pagestyle{fancy}

\titleformat{\section}{\bfseries\MakeUppercase}{\arabic{\thesection}}{1em}{}
\titleformat{\subsection}{\bfseries}{\arabic{\thesection}.\arabic{\thesubsection}}{1em}{}
\titleformat{\subsubsection}{\itshape}{\arabic{\thesection}.\arabic{\thesubsection}.\arabic{\thesubsubsection}}{1em}{}

\setlength{\parindent}{0cm}
\setlength{\parskip}{1em}

\captionsetup[figure]{labelfont=it, font=it}
\captionsetup[table]{labelfont={it,sc}, font={it,sc}}

\hypersetup{colorlinks, linkcolor = black, citecolor = black, urlcolor = black}
\urlstyle{same}



\begin{document}

\fancyfoot{}
\begin{center}
    \hfill \\
    \vspace{4in}
    {\bf\Huge CS457 Final Project \\}
    \vspace{2in}
    {\Large Soo-Hyun Yoo \\ March 18, 2015}
\end{center}

\newpage
\fancyfoot[C]{\sffamily Page \thepage\ of \pageref{LastPage}}

\section*{Source Files}

Source files are located in the {\tt src} subdirectory.

\begin{itemize}
    \item main.cpp
    \item smoke.vert
    \item smoke.frag
    \item glshader.hpp
    \item glshader.cpp
\end{itemize}


\section*{Explanation}

\subsection*{main.cpp}

The {\tt main} function in this file is the entry point to the program. It sets
up the output window size and some drawing flags before setting up the resize
and display functions which perform the actual object placement.

The vertex and fragment shaders are loaded using {\tt LoadShader} (found
online), which compiles and links the shaders to the program.

The scene itself contains a few geometric objects as seen through a uniform
fog.

\subsection*{smoke.vert}

We output some position vectors for lighting in the fragment shader.

\subsection*{smoke.frag}

We declare the input vectors and uniform variables.

After setting the color, we simulate fog by raycasting from each vertex to the
eye position. Although in the glman attempt of the same, noise and timer
functions preexisted and were trivial to use, implementing them here was not so
simple. A simplex noise function (again from the web) is included, though the
effect is not as pleasing as that of perlin noise due to the relatively sharper
cutoffs between light and dark.

\section*{Scope}

Due to poor timing, this is not quite the volumetric smoke outcome that was
promised in the project proposal. In order to render realistic volumetric
smoke, the smoke should be modeled as a particle system in a vector field
simulated by Navier-Stokes equations that model incompressible fluid flow.

The project as-is demonstrates that one-step raycasting in itself is fairly
simple. For the full effect showing realistic fluid flow, I envision that I can
set up two buffers in the main program and alternate between them to step
through each timestep in the fluid flow simulation.

\newpage
\section*{Results}

\begin{figure}[!h]
    \centering
    \includegraphics[width=1.0\textwidth]{img/smoke.png}
    \caption{Objects in uniform fog.}
\end{figure}

\begin{figure}[!h]
    \centering
    \includegraphics[width=1.0\textwidth]{img/simplexsmoke.png}
    \caption{Simplex noise added to fog.}
\end{figure}

\end{document}
