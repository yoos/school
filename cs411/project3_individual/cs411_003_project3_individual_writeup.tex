\documentclass[11pt,letterpaper]{article}
\usepackage[margin=1in]{geometry}
\usepackage{fancyhdr}
\usepackage[utf8]{inputenc}
\usepackage{palatino}
\usepackage{microtype}
\usepackage{hyperref}
\usepackage{graphicx}
\usepackage{lastpage}
\usepackage[hang,small]{caption}
\usepackage{titlesec}
\usepackage{amsmath,amssymb}
\usepackage{multirow}

\renewcommand{\headrulewidth}{0pt}
\fancyfoot{}
\fancyfoot[C]{\sf Page \thepage\ of \pageref{LastPage}}
\pagestyle{fancy}

\titleformat{\section}{\bfseries\Large}{\arabic{\thesection}}{1em}{}
\titleformat{\subsection}{\bfseries\large}{\arabic{\thesection}.\arabic{\thesubsection}}{1em}{}
\titleformat{\subsubsection}{\itshape}{\arabic{\thesection}.\arabic{\thesubsection}.\arabic{\thesubsubsection}}{1em}{}

\setlength{\parindent}{0cm}
\setlength{\parskip}{0.8em}

\captionsetup[figure]{labelfont=it,font=it}
\captionsetup[table]{labelfont={it,sc},font={it,sc}}

\hypersetup{colorlinks,
    linkcolor = black,
    citecolor = black,
    urlcolor  = black}
\urlstyle{same}



\begin{document}

Soo-Hyun Yoo \\
CS411 \\
Project 3 Group 03 Individual Writeup \\
19 May 2014


\section*{Project 3}

\subsection*{What do you think the main point of this assignment is?}

Implementing a block device prompted me to dig around the fs directory as well
as the block device drivers. The crypto part of the assignment was a bit
difficult because of the poorly documented API. Specifically, the data
structures were commented briefly, but none of the crypto functions were,
though they were at least consistently named. Fortunately, the large variety of
file system and block driver implementations provided enough precedence for us
to figure out how things should be put together.


\subsection*{How did you personally approach the problem? Design decisions,
algorithm, etc.}

I took a brief look through \verb|fs/ramfs| before resorting to using sbull
(Simple Block Utility for Loading Localities), which we found through Googling.
It provided us with a minimal block device driver which we could modify to
provide more debug information and include disk encryption.

The original \verb|memcpy| operations were replaced with calls to our own read
and write functions, each of which called a kerne debug print function in
addition to the \verb|crypto_cipher| encryption and decryption functions.


\subsection*{How did you ensure your solution was correct? Testing details, for
instance.}

Each read and write operation was accompanied by a printout of the operation
type and disk content to the kernel log. The encrypted content was also
printed.

We could therefore see that our modification of disk content (via \verb|dd| and
creation of text files) was valid.


\subsection*{What did you learn?}

As Kevin pointed out in class, given a large, undocumented codebase, it is
especially useful to refer to existing implementations to determine proper
usage.

I also note that whoever designed the crypto API did so through consistent
naming of the functions -- I wish more of the kernel was like this. This is not
to say that the API was well-documented, but it's fairly clear that the
designer had a model in mind.

\end{document}

