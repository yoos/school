\documentclass[11pt,letterpaper]{article}
\usepackage[margin=1in]{geometry}
\usepackage{fancyhdr}
\usepackage[utf8]{inputenc}
\usepackage{palatino}
\usepackage{microtype}
\usepackage{hyperref}
\usepackage{graphicx}
\usepackage{lastpage}
\usepackage[hang,small]{caption}
\usepackage{titlesec}
\usepackage{amsmath,amssymb}
\usepackage{multirow}
\usepackage{amsthm}

\usepackage{alltt}
\usepackage{float}
\usepackage{color}

\usepackage{balance}
\usepackage[TABBOTCAP, tight]{subfigure}
\usepackage{enumitem}
\usepackage{pstricks, pst-node, pst-tree}
\usepackage{geometry}

\renewcommand{\headrulewidth}{0pt}
\fancyhead{}
\fancyfoot{}
\fancyfoot[C]{\sf Page \thepage\ of \pageref{LastPage}}
\pagestyle{fancy}

\setlength{\parindent}{0cm}
\setlength{\parskip}{0.8em}

\hypersetup{colorlinks,
    linkcolor = black,
    citecolor = black,
    urlcolor  = black}
\urlstyle{same}



\begin{document}

\noindent {\bf Hugh McDonald \hfill CS 411 Group 03}

\noindent {\bf Bronson Mock}

\noindent {\bf Soo-Hyun Yoo}

\begin{center} \large \textbf{Project 4} \end{center}

\section*{Design}

\par We began this assignment by researching how the \verb|slob| (simple list
of blocks) memory management in Linux works. We utilizied resources available
online as well as in the book "Linux Kernel Development" by Robert Love
(namely, chapter 12).

\par We started by modifying the file \verb|slob.c| found in the \verb|mm|
folder. The given implementation of the \verb|slob| implemented a first-fit
algorithim. This algorithim chooses the first page that has enough space to fit
the request. This results in the fastest possible allocation but also results
in a high level of fragmentation.

\par We were asked to implement a best-fit algorithim for the \verb|slob|
algorithm. The best-fit algorithim we implemented searches through the
available pages. If it finds a perfect match, it will choose that one and end
the search. If no perfect match is found, when the algorithim reaches the end,
it will select the smallest page into which the request will fit. If the
request will not fit into any of the current pages, a new page will be
allocated for that request.

\par We decided to implement our code in \verb|#ifdef| blocks so that we could
toggle the use of best-fit and first-fit.

\section*{Testing}

\par We began by running the virtual machine with the first-fit algorithim and
then ran the best-fit algorithim. There was an apparent performance loss (i.e.,
system ran more slowly) with the best-fit algorithim. We then utilized the
kernel log to output the wasted space resulting from the best-fit algorithm
compared to what that space would have been with the first-fit algorithm.

\par Using a program that we wrote, we computed the optimization and efficiency
of the two algorithms. The program uses system calls that were added to
\verb|slob.c| to transfer information such as the memory actually used and the
memory that was claimed, which in turn allowed us to compare the fragmentation
of the best-fit and first-fit algorithims. The efficiency of the algorithims
was compared by timing how long it took each to allocate some chunk of memory.


\subsection*{Output From Kernel Log}

{\small
\begin{verbatim}
FIRST FIT DELTA: 4
BEST FIT DELTA: 0

FIRST FIT DELTA: 0
BEST FIT DELTA: 0

FIRST FIT DELTA: 4
BEST FIT DELTA: 0

FIRST FIT DELTA: 0
BEST FIT DELTA: 0

FIRST FIT DELTA: 1560
BEST FIT DELTA: 248

FIRST FIT DELTA: 734
BEST FIT DELTA: 36

FIRST FIT DELTA: 560
BEST FIT DELTA: 0

FIRST FIT DELTA: 0
BEST FIT DELTA: 0

FIRST FIT DELTA: 0
BEST FIT DELTA: 0

FIRST FIT DELTA: 2
BEST FIT DELTA: 0

FIRST FIT DELTA: 1694
BEST FIT DELTA: 414

FIRST FIT DELTA: 0
BEST FIT DELTA: 0

FIRST FIT DELTA: 1724
BEST FIT DELTA: 0

FIRST FIT DELTA: 0
BEST FIT DELTA: 0


\end{verbatim}
}

\end{document}

