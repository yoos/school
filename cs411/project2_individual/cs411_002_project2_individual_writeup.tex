\documentclass[11pt,letterpaper]{article}
\usepackage[margin=1in]{geometry}
\usepackage{fancyhdr}
\usepackage[utf8]{inputenc}
\usepackage{palatino}
\usepackage{microtype}
\usepackage{hyperref}
\usepackage{graphicx}
\usepackage{lastpage}
\usepackage[hang,small]{caption}
\usepackage{titlesec}
\usepackage{amsmath,amssymb}
\usepackage{multirow}

\renewcommand{\headrulewidth}{0pt}
\fancyfoot{}
\fancyfoot[C]{\sf Page \thepage\ of \pageref{LastPage}}
\pagestyle{fancy}

\titleformat{\section}{\bfseries\Large}{\arabic{\thesection}}{1em}{}
\titleformat{\subsection}{\bfseries\large}{\arabic{\thesection}.\arabic{\thesubsection}}{1em}{}
\titleformat{\subsubsection}{\itshape}{\arabic{\thesection}.\arabic{\thesubsection}.\arabic{\thesubsubsection}}{1em}{}

\setlength{\parindent}{0cm}
\setlength{\parskip}{0.8em}

\captionsetup[figure]{labelfont=it,font=it}
\captionsetup[table]{labelfont={it,sc},font={it,sc}}

\hypersetup{colorlinks,
    linkcolor = black,
    citecolor = black,
    urlcolor  = black}
\urlstyle{same}



\begin{document}

Soo-Hyun Yoo \\
CS411 \\
Project 2 Group 03 Individual Writeup \\
29 April 2014


\section*{Project 2}

\subsection*{What do you think the main point of this assignment is?}

Implementing the SSTF scheduler made clear the hardware issues that software
must deal with at some point. A secondary point may have been to further show
how the Linux kernel source has no naming scheme (i.e., struct elevator\_type
in the noop scheduler).


\subsection*{How did you personally approach the problem? Design decisions,
algorithm, etc.}

The SSTF scheduler needs to track requests in order of increasing or decreasing
sector number and track new requests behind (in opposite seek direction) the
head. This calls for two queues, so the data struct was updated accordingly.

The dispatch function was also updated to update the queues and change seek
direction after servicing requests. Finally, Linux's existing list sort
function was used to sort the queue after adding requests.


\subsection*{How did you ensure your solution was correct? Testing details, for
instance.}

The scheduler was tested to be correct by printing information regarding the
seek direction and the added and processed requests to the kernel log. We were
able to verify that every processed request was at a sector number higher or
lower than that of the previous request if the head was seeking up or down,
respectively.


\subsection*{What did you learn?}

Code modularity allows for easy extension of existing tools.

More importantly, I learned I don't want to program hardware drivers, at least
not at this level. Hardware is cool and I like robots, but I'm glad there are
other people working on things like disk access schedulers.

Again, Linux kernel developers aren't very organized or disciplined when it
comes to naming conventions. I can only presume it's nearly impossible to
refactor when the code is changing so frequently.

\end{document}

