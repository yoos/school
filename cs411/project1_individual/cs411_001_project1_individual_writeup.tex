\documentclass[11pt,letterpaper]{article}
\usepackage[margin=1in]{geometry}
\usepackage{fancyhdr}
\usepackage[utf8]{inputenc}
\usepackage{palatino}
\usepackage{microtype}
\usepackage{hyperref}
\usepackage{graphicx}
\usepackage{lastpage}
\usepackage[hang,small]{caption}
\usepackage{titlesec}
\usepackage{amsmath,amssymb}
\usepackage{multirow}

\renewcommand{\headrulewidth}{0pt}
\fancyfoot{}
\fancyfoot[C]{\sf Page \thepage\ of \pageref{LastPage}}
\pagestyle{fancy}

\titleformat{\section}{\bfseries\Large}{\arabic{\thesection}}{1em}{}
\titleformat{\subsection}{\bfseries\large}{\arabic{\thesection}.\arabic{\thesubsection}}{1em}{}
\titleformat{\subsubsection}{\itshape}{\arabic{\thesection}.\arabic{\thesubsection}.\arabic{\thesubsubsection}}{1em}{}

\setlength{\parindent}{0cm}
\setlength{\parskip}{0.8em}

\captionsetup[figure]{labelfont=it,font=it}
\captionsetup[table]{labelfont={it,sc},font={it,sc}}

\hypersetup{colorlinks,
    linkcolor = black,
    citecolor = black,
    urlcolor  = black}
\urlstyle{same}



\begin{document}

Soo-Hyun Yoo \\
CS411 \\
Project 1 Group 03 Individual Writeup \\
14 April 2014


\section*{Project 1}

\subsection*{What do you think the main point of this assignment is?}

To become acquainted with the structure of the Linux kernel source, how to
compile and deploy the kernel, and to learn about a few kernel debugging
methods.


\subsection*{How did you personally approach the problem? Design decisions,
algorithm, etc.}

The first thing I did was to look through the vanilla Linux kernel for the RT
scheduler implementations. A recursive diff showed that the given source code
was simply the vanilla kernel with a few lines removed, so determining what
parts (and how) to modify was easy.

To make this worthwhile, however, I went down a few rabbit holes of function
calls to see how the RT scheduler fit in the bigger picture.


\subsection*{How did you ensure your solution was correct? Testing details, for
instance.}

We wrote some simple C to spawn a few tasks under the FIFO scheduler. Each task
was made to run for a significant amount of time (i.e., count up to ULONG\_MAX)
while printing a few debug strings along the way. Under FIFO, we observed each
task complete before the next task was started, while under RR, the tasks
completed more concurrently.


\subsection*{What did you learn?}

In addition to how the FIFO and RR scheduling policies work and differ from the
normal policy, I learned about the runqueue struct and how the scheduler uses
it and other supporting objects.

\end{document}

