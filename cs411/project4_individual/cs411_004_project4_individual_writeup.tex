\documentclass[11pt,letterpaper]{article}
\usepackage[margin=1in]{geometry}
\usepackage{fancyhdr}
\usepackage[utf8]{inputenc}
\usepackage{palatino}
\usepackage{microtype}
\usepackage{hyperref}
\usepackage{graphicx}
\usepackage{lastpage}
\usepackage[hang,small]{caption}
\usepackage{titlesec}
\usepackage{amsmath,amssymb}
\usepackage{multirow}

\renewcommand{\headrulewidth}{0pt}
\fancyfoot{}
\fancyfoot[C]{\sf Page \thepage\ of \pageref{LastPage}}
\pagestyle{fancy}

\titleformat{\section}{\bfseries\Large}{\arabic{\thesection}}{1em}{}
\titleformat{\subsection}{\bfseries\large}{\arabic{\thesection}.\arabic{\thesubsection}}{1em}{}
\titleformat{\subsubsection}{\itshape}{\arabic{\thesection}.\arabic{\thesubsection}.\arabic{\thesubsubsection}}{1em}{}

\setlength{\parindent}{0cm}
\setlength{\parskip}{0.8em}

\captionsetup[figure]{labelfont=it,font=it}
\captionsetup[table]{labelfont={it,sc},font={it,sc}}

\hypersetup{colorlinks,
    linkcolor = black,
    citecolor = black,
    urlcolor  = black}
\urlstyle{same}



\begin{document}

Soo-Hyun Yoo \\
CS411 \\
Project 4 Group 03 Individual Writeup \\
2 June 2014


\section*{Project 4}

\subsection*{What do you think the main point of this assignment is?}

This assignment demonstrated the tradeoff that sometimes (often?) occurs
between execution speed and efficiency with resources.


\subsection*{How did you personally approach the problem? Design decisions,
algorithm, etc.}

I took a brief look through the SLOB and SLAB allocators to get a feel for how
they were organized, then we added our customizations within ifdef..endif
blocks.

New functions for the best-fit algorithm were added within ifdef..endif blocks
that examined all available blocks of memory, keeping track of blocks that
would result in the least amount of wasted space. This meant that we had to
look through the entire list of available blocks, but since the blocks are kept
track of in a singly-linked list, this was deemed the only way.


\subsection*{How did you ensure your solution was correct? Testing details, for
instance.}

With the best-fit algorithm, we printed to the kernel log the amount of wasted
memory that would have resulted if the first-fit algorithm had been used
compared to what actually resulting from having checked other available blocks
for a best fit. We were then able to verify that the best-fit wasted space was
always less than or equal to the amount wasted by first-fit. Furthermore, it
was evident based on the virtual machine's performance that the entire system
was slower, indicating lower efficiency.


\subsection*{What did you learn?}

It was interesting to see that the SLUB allocator was built on top of the SLAB
allocator, which in turn was built on the SLOB allocator. This demonstrates
code reuse and some backwards-compatibility in the Linux kernel.

\end{document}

