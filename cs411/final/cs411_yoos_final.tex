\documentclass[12pt,letterpaper]{article}
\usepackage[margin=1in]{geometry}
\usepackage{fancyhdr}
\usepackage[utf8]{inputenc}
\usepackage{palatino}
\usepackage{microtype}
\usepackage{hyperref}
\usepackage{graphicx}
\usepackage{lastpage}
\usepackage[hang,small]{caption}
\usepackage{titlesec}
\usepackage{amsmath,amssymb}
\usepackage{multirow}

\renewcommand{\headrulewidth}{0pt}
\fancyfoot{}
\fancyfoot[C]{\sf Page \thepage\ of \pageref{LastPage}}
\pagestyle{fancy}

\titleformat{\section}{\bfseries\Large}{\arabic{\thesection}}{1em}{}
\titleformat{\subsection}{\bfseries\large}{\arabic{\thesection}.\arabic{\thesubsection}}{1em}{}
\titleformat{\subsubsection}{\itshape}{\arabic{\thesection}.\arabic{\thesubsection}.\arabic{\thesubsubsection}}{1em}{}

\setlength{\parindent}{0cm}
\setlength{\parskip}{0.8em}

\captionsetup[figure]{labelfont=it,font=it}
\captionsetup[table]{labelfont={it,sc},font={it,sc}}

\hypersetup{colorlinks,
    linkcolor = black,
    citecolor = black,
    urlcolor  = black}
\urlstyle{same}



\begin{document}

Soo-Hyun Yoo \\
CS411 \\
Final Exam \\
13 June 2014


\section*{Analysis of the Android x86 Build Tree}

For the most part, the Android kernel appears to be essentially the same as the
standard Linux kernel. Consequently, I could not find many differences between
the two kernels related to the concepts covered in the four assignments. For
example, the available I/O schedulers, SLOB/SLAB/SLUB allocators, basic file
I/O operators, and the crypto API are the same. Hardware differences are
accounted for in the architecture-specific drivers. Any modifications thus
exist in order to incorporate the larger additions made by Android.

Android's additions to the Linux kernel exist, as will be illustrated in the
two subsections below, to make the same programs work on many different
hardware platforms with limited resources while providing top performance to
the user.

Towards that effort, Android employs a Java virtual machine dubbed Dalvik which
runs all applications. Because processes run within this VM, applications can
be easily written to run on any architecture supported by Android.
Additionally, Android includes a new file system type called YAFFS2 (Yet
Another Flash File System v2), which was designed specifically for NAND-based
flash storage common on mobile devices.


\subsection*{Process Scheduler}

Although the CFS and the FIFO and RR schedulers are the same in both the
Android and Linux standard kernels (i.e., first-in-first-out linked list queue,
allotting timeslices to processes to determine when to switch context), the
Android kernel prioritizes user experience over everything else. This means
that applications in the foreground are given higher priority over those in the
background and are able to preempt those in the background.

There are also more subtle changes in \verb|kernel/kernel/sched.c|, such as
Android's use of jiffies instead of comparing CPU ticks to a calculated
\verb|LOAD_FREQ| for load balancing (presumably for increased accuracy?),
reporting CPU power consumption, and refactoring the scheduler to make it
available in userspace.


\subsection*{Shared Memory}

The Android kernel implements a special shared memory system in
\verb|kernel/mm/ashmem.c| that helps the kernel and the applications above it
function smoothly on lower-memory devices by sharing common resources. In
addition to having a reference counter to keep track of whether or not the
shared memory space is still being used by any processes, ashmem allows the
kernel to release the shared memory and provides bindings for processes to
reclaim this memory if it disappears. This is in contrast to the standard
shared memory system, which does not allow shared memory to be freed by other
processes.

By ``unpinning'' a portion of memory, a process indicates to the kernel that
that portion of memory is free to be claimed for other uses if the kernel feels
the system is low on memory. The kernel commandeers this unpinned memory
according to a FIFO linked list as defined in \verb|ashmem.c| (referred to as
LRU, or Least Recently Used) and determines the amount of memory to be
commandeered using ashmem's shrinker, also defined in \verb|ashmem.c|:

{\small
\begin{verbatim}
static int ashmem_shrink(struct shrinker *s, struct shrink_control *sc)
{
    struct ashmem_range *range, *next;

    /* we might recurse into filesystem code, so bail out if necessary */
    if (sc->nr_to_scan && !(sc->gfp_mask & __gfp_fs))
        return -1;
    if (!sc->nr_to_scan)
        return lru_count;

    mutex_lock(&ashmem_mutex);
    list_for_each_entry_safe(range, next, &ashmem_lru_list, lru) {
        struct inode *inode = range->asma->file->f_dentry->d_inode;
        loff_t start = range->pgstart * page_size;
        loff_t end = (range->pgend + 1) * page_size - 1;

        vmtruncate_range(inode, start, end);
        range->purged = ashmem_was_purged;
        lru_del(range);

        sc->nr_to_scan -= range_size(range);
        if (sc->nr_to_scan <= 0)
            break;
    }
    mutex_unlock(&ashmem_mutex);

    return lru_count;
}
\end{verbatim}
}

Ashmem is made accessible through a userspace API defined in \\
\verb|system/core/libcutils/ashmem-dev.c|. Each process keeps enough memory
pinned to safely store its state, so they are able to reallocate resources as
the need arises later.

The actual sharing of the memory occurs via file descriptors, which in turn are
shared between processes using another Android-specific IPC service called the
binder. Once mmap'ed, the shared memory space can be utilized using file I/O
operations. This means the memory may be shared between processes that do not
share a parent-child relationship. The ashmem read function, for example:

This ability for processes to temporarily and nondestructively relinquish
memory is an important feature for Android due to its inability to use swap
space and its need to remain responsive for the user despite having less
available memory compared to a desktop system.

Finally, also unlike standard shared memory, ashmem dies with the process. With
a kernel that commandeers resources away from processes left and right and even
kills them to free up memory, usage of the standard shared memory system would
result in consistent memory leaks that would force the user to reboot the
device.


\subsection*{Conclusion}

Besides the above Android has added small but significant features such as USB
debugging and kernel logging to RAM in order to ease development for embedded
high-performance systems that are today's mobile devices. Much attention seems
to have been paid to prioritizing user comfort and overall smoothness of the
system. It's inspiring to see that even with such radical changes to the
surface appearance of the operating system, the changes made to the inner
workings are not all that extensive (albeit clever).

\end{document}

