\documentclass[12pt,letterpaper]{article}
\usepackage[margin=1in]{geometry}
\usepackage{fancyhdr}
\usepackage[utf8]{inputenc}
\usepackage{palatino}
\usepackage{microtype}
\usepackage{hyperref}
\usepackage{graphicx}
\usepackage{lastpage}
\usepackage[hang,bf,small]{caption}
\usepackage{titlesec}
\usepackage{amsmath,amssymb,amsthm}

\renewcommand{\headrulewidth}{0pt}
\fancyfoot{}
\fancyfoot[C]{\thepage}
\pagestyle{fancy}

% \titleformat{\section}{\bfseries\MakeUppercase}{\arabic{\thesection}}{1em}{}
% \titleformat{\subsection}{\bfseries}{\arabic{\thesection}.\arabic{\thesubsection}}{1em}{}
% \titleformat{\subsubsection}{\itshape}{\arabic{\thesection}.\arabic{\thesubsection}.\arabic{\thesubsubsection}}{1em}{}

\setlength{\parindent}{0cm}
\setlength{\parskip}{1em}

% \captionsetup[figure]{labelfont=it,font=it}
% \captionsetup[table]{labelfont={it,sc},font={it,sc}}

\hypersetup{colorlinks,
    linkcolor = black,
    citecolor = black,
    urlcolor  = black}
\urlstyle{same}


\begin{document}

Soo-Hyun Yoo \\
CS311 \\
Homework 3 \\
22 October 2012


\section*{Design}

Once compiled, {\tt myar} will run in a manner similar to that of the standard
UNIX command {\tt ar}. The program will be written in C and will support the
following options:

\begin{itemize}
	\item q: quickly append named files to archive
	\item x: extract named files
	\item t: print a concise table of contents of the archive
	\item v: print a verbose table of contents of the archive
	\item d: delete named files from archive
	\item A: quickly append all ``regular'' files in the current directory
		(except the archive itself)
\end{itemize}

Each option will be paired up with a function in the code, and each function
may use smaller helper functions to find files in the archive as necessary.
{\tt myar} will take only one option at a time, and a switch statement will run
the appropriate function.

Since deletion of a file from the archive will leave a gap in the archive, any
delete operation will require the program to overwrite the old file with a new
file without any gaps.

Since file I/O is expensive, the number of passes through the archive file will
be minimized by first determining which of the files requested for deletion
actually exist in the archive, in order of appearance, then parsing through the
archive file and appending each file to a new archive file if the file is not
in the deletion list.

Read/write operations are done 16 bytes at a time in order to keep memory usage
low.

File extraction will check for the existence of a file in the target directory
and prompt the user before overwriting.

The q, x, and d operations can take multiple files as inputs.


\section*{Worklog}

\subsection*{Oct. 15}

\begin{itemize}
	\item Initial commit.
\end{itemize}

\subsection*{Oct .22}

\begin{itemize}
	\item Add some (undefined) functions we might need.
	\item Use C comments.
	\item Add switch statement for key.
\end{itemize}

\subsection*{Oct. 23}

\begin{itemize}
	\item Add .gitignore.
	\item Add TeX file.
	\item Start writeup of Design section.
	\item Program builds.
	\item Add Makefile.
	\item Pass archive name to functions.
	\item Successfully append file to archive.
	\item Properly create and check for archive file.
	\item Make writeToFile function.
\end{itemize}

\subsection*{Oct. 24}

\begin{itemize}
	\item Implement header writing in append().
	\item Append works.
	\item Implement get() and extract() functions.
	\item Minor formatting edit.
	\item Rename get() to getHeader().
	\item Make extract() prompt user if file already exists.
	\item Fix incorrect design description.
	\item Multi-input append and extract in a mostly working state.
	\item Finish working on extract function.
	\item Minor change.
	\item Implement table of contents functions.
	\item Tidy things up.
\end{itemize}

\subsection*{Oct. 25}

\begin{itemize}
	\item Minor fixes.
	\item Remove debug printf.
	\item Implement delete function.
	\item Add condition to make extract() more efficient.
	\item Fix filename matching.
	\item Remove obsolete todo comments.
	\item Remove unnecessary argument.
	\item Implement directory listing.
	\item Use strncmp instead of memcmp.
\end{itemize}

\subsection*{Oct. 26}

\begin{itemize}
	\item Start implementing function for -A option.
	\item Fix segfaulting problem by moving char buffer outside of functions.
	\item Check that all functions work.
	\item Fix name comparison so we correctly find files in archive.
	\item Comment out debug printfs.
	\item Update design section.
	\item Add comment.
	\item Do not compile TeX file in default.
	\item Minor changes.
	\item Add README.
	\item Use Intel compilers.
	\item Fix miscellaneous problems that were not apparent while using gcc.
	\item Minor change.
	\item Add worklog.
	\item Itemize the items!
\end{itemize}


\section*{Challenges}

The functions were fairly straightforward to implement, but correctly passing
pointers around proved to be a delicate task.

I spent half of my time on this project debugging something that proved to be
a nearly one-line fix. The buffer of size 16 used frequently in the functions
in this program used to be defined within the function. This caused a segfault,
depending a strange mix of conditions, including the length of filenames and
the cumulative size of all input files.

Eventually, I took note of the fact that valgrind reported some ``bad
permissions for mapped region,'' and determined that as a function exited, the
buffer it was using to read or write data may have been deallocated before the
file I/O operation had finished, resulting in the segfault. Declaring the
buffer in global scope solved this issue. (Funnily enough, syncfs() did not
help.)


\section*{Answers to questions}

This assignment helped me familiarize myself with file I/O operations and
string manipulation. The bug mentioned in the previous section made me
reexamine how pointers work (particularly, double and triple pointers) and
reminded me that I should be better friends with gdb and valgrind.

The program was checked for correctness using various inputs files of varying
sizes. For example:

\begin{itemize}
	\item {\tt ./myar q arfile f1}, where {\tt f1} is a small text file.
	\item {\tt ./myar q arfile f1 f1 f1 f1 f1 f1}
	\item {\tt ./myar q arfile myar}
	\item {\tt ./myar q arfile f2}, where {\tt f2} is a large (1 MB) text file.
	\item {\tt ./myar d arfile f1 f2}
	\item {\tt ./myar A arfile}
	\item {\tt ./myar x f1 f2 myar}
\end{itemize}

\dots and so on. The archive file itself was examined manually, in addition to
being tested for compatibility with the standard UNIX ar command for q, d, x,
t, and v.

As stated, I learned how to use file I/O system calls, how to pass around double pointers such as {\tt argv}, and that scope is important.

\end{document}
