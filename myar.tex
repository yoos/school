\documentclass[12pt,letterpaper]{article}
\usepackage[margin=1in]{geometry}
\usepackage{fancyhdr}
\usepackage[utf8]{inputenc}
\usepackage{palatino}
\usepackage{microtype}
\usepackage{hyperref}
\usepackage{graphicx}
\usepackage{lastpage}
\usepackage[hang,bf,small]{caption}
\usepackage{titlesec}
\usepackage{amsmath,amssymb,amsthm}

\renewcommand{\headrulewidth}{0pt}
\fancyfoot{}
\fancyfoot[C]{\thepage}
\pagestyle{fancy}

% \titleformat{\section}{\bfseries\MakeUppercase}{\arabic{\thesection}}{1em}{}
% \titleformat{\subsection}{\bfseries}{\arabic{\thesection}.\arabic{\thesubsection}}{1em}{}
% \titleformat{\subsubsection}{\itshape}{\arabic{\thesection}.\arabic{\thesubsection}.\arabic{\thesubsubsection}}{1em}{}

\setlength{\parindent}{0cm}
\setlength{\parskip}{1em}

% \captionsetup[figure]{labelfont=it,font=it}
% \captionsetup[table]{labelfont={it,sc},font={it,sc}}

\hypersetup{colorlinks,
    linkcolor = black,
    citecolor = black,
    urlcolor  = black}
\urlstyle{same}


\begin{document}

Soo-Hyun Yoo \\
CS311 \\
Homework 3 \\
22 October 2012


\section*{Design}

Once compiled, {\tt myar} will run in a manner similar to that of the standard
UNIX command {\tt ar}. The program will be written in C and will support the
following options:

\begin{itemize}
	\item -q: quickly append named files to archive
	\item -x: extract named files
	\item -t: print a concise table of contents of the archive
	\item -v: print a verbose table of contents of the archive
	\item -d: delete named files from archive
	\item -A: quickly append all "regular" files in the current directory
		(except the archive itself)
\end{itemize}

Each option will be paired up with a function in the code, and each function may use smaller helper functions to find files in the archive as necessary. {\tt myar} will take only one option at a time, and a switch statement will run the appropriate function.

Since deletion of a file from the archive will leave a gap in the archive, any delete operation will require the program to overwrite the old file with a new file without any gaps.


\section*{Worklog}


\section*{Challenges}


\section*{Answers to questions}



\end{document}
