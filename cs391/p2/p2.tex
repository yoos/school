\documentclass[12pt,letterpaper]{article}
\usepackage[margin=1in]{geometry}
\usepackage{fancyhdr}
\usepackage[utf8]{inputenc}
\usepackage{palatino}
\usepackage{microtype}
\usepackage{hyperref}
\usepackage{graphicx}
\usepackage{lastpage}
\usepackage[hang,small,margin=1in]{caption}
\usepackage{titlesec}
\usepackage[authoryear,colon,sort&compress]{natbib}

\renewcommand{\headrulewidth}{0pt}
\fancyfoot{}
\fancyfoot[C]{\sffamily Page \thepage\ of \pageref{LastPage}}
\pagestyle{fancy}

\titleformat{\section}{\bfseries\MakeUppercase}{\arabic{\thesection}}{1em}{}
\titleformat{\subsection}{\bfseries}{\arabic{\thesection}.\arabic{\thesubsection}}{1em}{}
\titleformat{\subsubsection}{\itshape}{\arabic{\thesection}.\arabic{\thesubsection}.\arabic{\thesubsubsection}}{1em}{}

\setlength{\parindent}{0cm}
\setlength{\parskip}{1em}

\captionsetup[figure]{labelfont=it, font=it}
\captionsetup[table]{labelfont={it,sc}, font={it,sc}}

\hypersetup{colorlinks, linkcolor = black, citecolor = black, urlcolor = black}
\urlstyle{same}


\begin{document}

\fancyfoot{}
\begin{center}
    \hfill \\
    \vspace{4in}
    {\bf\Huge CS391 Paper 2 \\}
    \vspace{2in}
    {\Large Soo-Hyun Yoo \\ May 13, 2015}
\end{center}

\newpage
\fancyfoot[C]{\sffamily Page \thepage\ of \pageref{LastPage}}

\section*{Open Source and Intellectual Property Rights in the United States}

% Explain how the concept of Open Source fits into the 
% legal framework of intellectual property rights as defined 
% in the United States.

The law is rarely ahead of emerging technologies. Laws protecting intellectual
property (IP) rights are no exception -- in the wake of the rapidly evolving
software industry, traditional IP protection methods are proving to be
insufficient or outright inappropriate. The failure of the IP protection system
to meet the needs of the software industry has spurred the open source
movement. We examine the open source movement, its compatibility with existing
IP laws, and discuss some ways of reconciliation between the two.

\subsection*{Defining Open Source}

% 1. Define what open source software is

Like other material creations of the mind, software can be licensed for a fee
to benefit the creators of the software. Consequently, governments have
extended IP rights to software, but such proprietary software is not without
downsides~\citep[p. 200]{quinn2015}.

Richard Stallman, one of the foremost critics of proprietary software, points
out that the copyright system has not adapted to the digital era and is
infringing on our liberties. In addition, instead of incentivizing software
creators to work hard and improve their software to outperform their
competitors, closed-source software hinders progress~\citep[p. 201]{quinn2015}.

This concern is a major driving force behind the open source movement. Open
source software ``can be freely used, changed, and shared...  by anyone,''
which means it typically includes the source code in its distribution (or make
it easily accessible to the public). Furthermore, the~\citet{osi2015} dictates
that an open source product cannot discriminate ``against persons or groups...
fields of endeavor,'' be limited to a certain product or technology, or
restrict other software included within. By adopting these requirements,
a product under an open source license hopes to promote fast, long-lasting
progress, and freedom of use, even for commercial purposes~\citep{osi2015}.


\subsection*{Types of Intellectual Property}

% 2. Describe how the different forms of intellectual 
% property apply to Open Source projects

In the U.S. today, IP can be protected in four different ways: by trade
secrets, trademarks, patents, or copyrights~\citep[p. 169]{quinn2015}.

A trade secret protects IP by confidentiality. It is the only method of the
four types of IP protection whose enforcement is entirely up to the entity that
``owns'' the IP~\citep[p. 169]{quinn2015}; if the IP becomes public knowledge
by any means, the entity loses all rights to the IP due to its no longer being
a secret. However, it is still illegal to steal trade secrets per the 1996
Economic Espionage Act, which condemns anyone who knowingly targets or acquires
a trade secret to benefit anyone or any foreign entity that is not the owner of
the trade secret~\citep{fbi2015} if the owner has taken significant steps to
hide the IP from the public.

The government grants a trademark or servicemark to a company as a right to use
a word, picture, or another kind of mark to identify its product or service and
to prevent other companies from using the same mark. This allows companies to
build a brand image, which can give consumers confidence in the products they
buy. Companies protect their trademarks from being used as nouns or verbs (lest
they lose it) by contacting those who misuse the trademarks~\citep[p.
170]{quinn2015}.

IP can be patented through the government, giving the owner exclusive rights to
make, use, or sell the product and to prevent others from doing so for 20
years. In exchange, the patent application, containing a detailed description
of the product, is made public for others to improve upon~\citep[p.
170]{quinn2015}.

A copyright provides the author of a written work the right to control the
reproduction, distribution, display, performance, and production of derivatives
of the work. Copyright industries such as the movie, music, and software
industries comprise over 6 percent of U.S. GDP. Although copyright lasted only
28 years when the first Copyright Act was passed, various amendments have
drastically increased the duration and breadth of coverage of copyrights,
leading some to criticize the Supreme Court of overprotecting private rights
and Congress of overstepping its constitutional power to ``grant exclusive
rights to authors for `limited times' ''~\citep[p. 172]{quinn2015}.


\subsection*{Open Source Software Compatibility with Intellectual Property Rights}

% 3. Describe how the concept of intellectual property 
% rights might be at odds with Open Source

Protection of IP rights is ultimately meant to encourage new ideas and
improvement of old technologies by rewarding hard workers with recognition and
monetary benefits. Depending on the intended use of the software and the
particular IP protection type, the same protections may not necessarily benefit
society and thus be at odds with open source software.

Just like any proprietary design, software can be kept as a trade secret within
a company. If it wishes, the company may share partial source code, keeping
strategic sections of a software product a trade secret while still benefiting
society by sharing the source of core functions of the product. The company may
also decide when it no longer needs to keep it a secret, and third parties that
manage to reverse engineer or otherwise legally obtain such trade secrets to
incorporate into an open source project can do so without legal repercussions.
In that sense, trade secrets may be compatible with open source and be able to
adapt to changing patterns in industry.

On the other hand, keeping software a trade secret completely hides the work
from public view and confines benefits to a single individual or company. This
runs counter to the ultimate goal of both IP protection and the open source
movement, which is to speed software progress. Thus, it is not a good idea to
keep software as trade secrets.

Trademarks are useful to have for all kinds of products, proprietary or open,
software or not. When choosing a product or service to use, people should be
guaranteed by its name that it is the same product they have used in the past
rather than a potentially inferior clone or an irrelevant product altogether.
Even the~\citet{osi2015} has trademarked its name and logo so that programmers
can work around the same initiative.

Software patents could be considered marginally better than trade secrets since
the software design is made public for others to improve upon. However, giving
the creator exclusive rights to software use and sale obviously conflicts with
the goal of rapid and wide circulation of new technology. Given that entire
languages spring up in the span of a few years, let alone many different
variations of similar products and services, locking away the rights to a piece
of software under a patent for 20 years is not compatible at all.

Copyright, however, is absolutely compatible with open source software, given
that an open source license itself is an application of copyright law.
Traditionally, copyrights have been used to deter others from reproducing or
adapting written works without permission, thereby limiting others' freedom to
do as they wish with the product. An open source license enforces product use
in the opposite direction -- anyone wishing to use or reproduce the software is
allowed to do so but is often required to retain the same open source license
so that others can benefit~\citep{osi2015}, all the while preserving
attribution of the work to the author.

On the whole, open source software has already demonstrated its capability to
coexist with proprietary software. MySQL, for example, is available both
publicly under the GPL and commercially under a more traditional
license~\citep[p. 2]{pugatch2008}. The open source movement has also spurred
various groups to form open standards, under which companies have developed and
marketed proprietary products (e.g., USB devices, firewalls)~\citep[p.
3]{pugatch2008}. Even ``open hardware'' projects are commonly seen in the
hobbyist community. Coupled with open source software, they empower the average
citizen to create new products, which many decide to monetize through online
markets such as Kickstarter and Tindie. Open source and IP rights appear to be
getting along just fine.

\bibliographystyle{apa}
\bibliography{p2.bib}

\end{document}
