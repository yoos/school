\documentclass[12pt,letterpaper]{article}
\usepackage[margin=1in]{geometry}
\usepackage{fancyhdr}
\usepackage[utf8]{inputenc}
\usepackage{palatino}
\usepackage{microtype}
\usepackage{hyperref}
\usepackage{graphicx}
\usepackage{lastpage}
\usepackage[hang,small,margin=1in]{caption}
\usepackage{titlesec}
\usepackage[authoryear,comma,sort&compress]{natbib}

\renewcommand{\headrulewidth}{0pt}
\fancyfoot{}
\fancyfoot[C]{\sffamily Page \thepage\ of \pageref{LastPage}}
\pagestyle{fancy}

\titleformat{\section}{\bfseries\MakeUppercase}{\arabic{\thesection}}{1em}{}
\titleformat{\subsection}{\bfseries}{\arabic{\thesection}.\arabic{\thesubsection}}{1em}{}
\titleformat{\subsubsection}{\itshape}{\arabic{\thesection}.\arabic{\thesubsection}.\arabic{\thesubsubsection}}{1em}{}

\setlength{\parindent}{0cm}
\setlength{\parskip}{1em}

\captionsetup[figure]{labelfont=it, font=it}
\captionsetup[table]{labelfont={it,sc}, font={it,sc}}

\hypersetup{colorlinks, linkcolor = black, citecolor = black, urlcolor = black}
\urlstyle{same}

\renewcommand{\citep}[1]{(\cite{#1})}


\begin{document}

\fancyfoot{}
\begin{center}
    \hfill \\
    \vspace{4in}
    {\bf\Huge CS391 Paper 1 \\}
    \vspace{2in}
    {\Large Soo-Hyun Yoo \\ April 24, 2015}
\end{center}

\newpage
\fancyfoot[C]{\sffamily Page \thepage\ of \pageref{LastPage}}

\section*{The Edward Snowden Case Examined}

\subsection*{Case Summary}

Edward Snowden is famous for his revelation of the vast reach of the NSA into
the lives of not only foreigners, but particularly of American citizens. As
a former subcontractor working for the NSA's cryptological center in Hawaii,
Snowden became disillusioned by the NSA's clandestine and invasive surveillance
practices. After receiving no support from his superiors and watching the
newly-incumbent President Obama advance the very same practices, Snowden took
an unpaid leave of absence, taking with him a large number of classified files
he had been collecting over the preceding few years. He contacted Laura Poitras
and Glenn Greenwald, which led to The Guardian's report on the NSA's secret
wiretapping of Americans and the PRISM program. Many more reports would
follow~\citep{harding2014}.

Some of the most alarming practices revealed by Snowden include wiretapping of
top government officials all around the world and unwarranted monitoring of
personal communications within the U.S. Snowden was able to show the NSA and
GCHQ (the British counterpart to the NSA) were eavesdropping on Italian
communications. According to the Snowden documents, the NSA has also been
responsible for more than 61,000 hacking operations worldwide, spied on
European Union offices, intercepted German Chancellor Angela Merkel's phone
calls along with those of 35 other world leaders, and collects and stores more
than 200 million text messages per day globally. Snowden has been charged with
``theft of government property, unauthorised communication of national defence
information and willful communication of classified communications
intelligence''~\citep{bbc2014}. He currently lives under asylum in
Russia~\citep{rusbridger2014}.


\subsection*{Snowden's Actions Justified Under Act Utilitarianism}

The Snowden revelations has sparked intense debate on surveillance around the
world and has induced real changes in companies to increasing their security
measures and the U.S. government admitting to and scaling back some of its
surveillance programs. Let us consider the happiness of the global and American
populations under Act Utilitarianism.

Surveillance of personal communication is a violation of our right to privacy.
Given that the NSA exists to serve the interests of the U.S. population at
best, it is simple to determine that a decrease in NSA surveillance is an
absolute positive for anyone outside of the U.S.

The case for the American populace is a bit more entangled. Per its namesake,
the NSA exists to ``gain a decision advantage for the Nation and our allies
under all circumstances.'' Under Executive Order 12333, issued 1981, the
director of the NSA is tasked to ``collect (including through clandestine
means), process, analyze, produce... information... to support national and
departmental missions.'' Particularly, it was amended in 2008 prior to the
Snowden revelations in order to ``maintain or strengthen privacy and civil
liberties protections''~\citep{nsa2015}.

If this information collection and analysis is performed solely on
non-Americans, we could argue there is no net negative effect on the happiness
of Americans. However, considering that American civilians have been the target
of the same surveillance programs used by the NSA to spy on foreign bodies, we
must admit that the NSA has violated our own right to privacy along with that
of the rest of the world. From a human rights standpoint, this indicates an
overall increase in happiness for the American public as a result of the
Snowden revelations.

We can also consider the original goal of the NSA in collecting this
information. The NSA's domestic spying program was implemented by the Bush
administration following the 9/11 attacks in an effort to uncover terrorist
plots within the U.S.~\citep{eff2015}. The Snowden files revealed that the NSA
was targeting American citizens without warrant and with little oversight. Such
warrantless and broad surveillance discourages people from exercising their
freedoms, eroding our democracy. Accepting democracy and freedom as generally
good, Snowden's whistleblowing has resulted in a net increase in happiness for
the American people.

Furthermore, Snowden was extremely careful in making this information available
to the public. Unlike Julian Assange's Wikileaks leak from a few years prior,
Snowden release no file directly, instead opting to let competent journalists
take time to sift through the secret documents and publish them
responsibly~\citep{rusbridger2014}. It would have been pointless for our right
to privacy first violated by the government, then by the entire world. By
refraining from indiscriminately publishing the documents to the Web, Snowden
avoided causing significant damage to American lives.


\subsection*{Government's Actions Justified Under Kantianism}

With perfect information, the government can provide perfect protection for its
people. By maximizing the reach of the NSA both within and outside the U.S., it
could hope to gain this perfect information to best serve our national security
interests. For example, the 9/11 attack could have been prevented if we had had
prior knowledge. Perfect information would have provided us with this
knowledge. If we can prevent even one more terrorist attack with knowledge
gained from a more extensive surveillance network, such a network is worth
pursuing.

Furthermore, the surveillance targets are all those suspected to be affiliated
with a terrorist group. Given a real chance of a terrorist organization
operating in or outside the U.S. who may wish harm to the U.S. and its people,
increased knowledge about such targets and their motives is especially
paramount in protecting our country.

Another point worth considering is the fact that the NSA monitors only metadata
to build a network of information about suspected terrorist targets before
actually looking at the data itself. For example, the simply monitoring
a person's phone calls and emails is not equivalent to outright spying, as the
content of the communications is scrutinized only if there is reason to believe
the person is involved with terrorist activities. This allows the NSA to
selectively monitor real terrorist threats without causing undue violation of
the average person's right to privacy~\citep{nsa2013}.

Finally, the NSA ``has an internal oversight and compliance framework to... act
consistently with the law and with NSA and U.S. intelligence community policies
and procedures.'' A purely beneficial founding cause, motive, and mode of
operation justifyWith  the NSA's surveillance methods.



% Using Ethical frameworks to argue pro’s and cons.  Study the actions of
% Edward Snowden and his collection and subsequent leak of classified
% information regarding US intelligence operations.

% a. Give a brief summary of the facts of the case, as well as a chronology of
% events up to Mr. Snowden’s flight to Russia.

% b. Briefly summarize the most important revelations from the information Mr.
% Snowden leaked.

% c. Pick a workable ethical framework as presented in this class, and use it
% to justify Mr. Snowden’s actions. This does not have to match with Mr.
% Snowden’s own justifications, but must match the facts of the case.

% d. Pick a workable ethical framework as presented in this class, and use it
% to justify the U.S. government’s actions leading up to, and following the
% leak. This justification does not have to match the official U.S. position,
% but needs to match the facts of the case.

% e. You will be graded on the internal logic of your arguments, as well as
% your adherence to the ethical frameworks you chose.

\bibliographystyle{apa}
\bibliography{p1.bib}

\end{document}
