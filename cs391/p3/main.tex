\documentclass[12pt,letterpaper]{article}
\usepackage[margin=1in]{geometry}
\usepackage{fancyhdr}
\usepackage[utf8]{inputenc}
\usepackage{palatino}
\usepackage{microtype}
\usepackage{hyperref}
\usepackage{graphicx}
\usepackage{lastpage}
\usepackage[hang,small,margin=1in]{caption}
\usepackage{titlesec}
\usepackage[authoryear,colon,sort&compress]{natbib}

\renewcommand{\headrulewidth}{0pt}
\fancyfoot{}
\fancyfoot[C]{\sffamily Page \thepage\ of \pageref{LastPage}}
\pagestyle{fancy}

\titleformat{\section}{\bfseries\MakeUppercase}{\arabic{\thesection}}{1em}{}
\titleformat{\subsection}{\bfseries}{\arabic{\thesection}.\arabic{\thesubsection}}{1em}{}
\titleformat{\subsubsection}{\itshape}{\arabic{\thesection}.\arabic{\thesubsection}.\arabic{\thesubsubsection}}{1em}{}

\setlength{\parindent}{0cm}
\setlength{\parskip}{1em}

\captionsetup[figure]{labelfont=it, font=it}
\captionsetup[table]{labelfont={it,sc}, font={it,sc}}

\hypersetup{colorlinks, linkcolor = black, citecolor = black, urlcolor = black}
\urlstyle{same}


\begin{document}

\fancyfoot{}
\begin{center}
  \hfill \\
  \vspace{4in}
  {\bf\Huge CS391 Paper 3 \\}
  \vspace{2in}
  {\Large Soo-Hyun Yoo \\ May 29, 2015}
\end{center}

\newpage
\fancyfoot[C]{\sffamily Page \thepage\ of \pageref{LastPage}}

\section*{Internet Addiction}

I made the most striking realization before the first day was out: literally
all of my distractions are on the Internet. In hindsight, this should have been
obvious, but having had the Internet be a constant all my life prevented me
from categorizing the distractions as such. Usually I would absentmindedly
check various news sites, Hacker News, and Imgur, all of which I missed.
Without these, I had to find other things to do.

I had many small chores to do around the house I had been putting off.
I vacuumed, mowed the overgrown lawn, and cleaned my garage workstation that
had become a pile of boxes. I also spent a lot of time poking around my car,
which has a neverending list of things to be fixed, cleaned and lubricated my
bike, and organized cluttered directories on my laptop.

To help myself stay off the Internet, I disabled my laptop's wireless card and
put my phone in airplane mode. Whereas normally, I would pull up a browser
window to check my favorite news sites rather than complete chores, having the
Internet be a few steps away encouraged me to commit to the task at hand. This
also demonstrated that smartphones lose a huge part of their utility without an
Internet connection. While it is true I've become accustomed to my phone being
much more than just a phone, it was still disorienting to use it as simply
a phone rather than a media device.

Additionally, instead of checking every two minutes for updates to the bus
arrival time, I had to leave a few minutes early to catch the bus at the
scheduled time. While I could not maintain the razor-thin tolerances in my
daily schedule (it was unhealthy anyway), I found myself keeping track of my
obligations for the day a bit more clearly a few times throughout the day
rather than compulsively checking my online calendar. In the long run, I would
have had to find another way of scheduling events -- I would probably revert to
the simple text file todo system I used once in the past.

The world is quiet without Pandora. It felt a bit like mountain biking alone in
the woods for two days minus the biking.

I found it most disruptive to lack access to the Internet when I tried to do
some of my usual firmware work. I was not surprised to find that without Stack
Overflow to augment my programming knowledge, I was lost when it came to
cryptic linker errors. Perhaps having a GCC C++ compiler handbook and knowing
the internals of the Boost libraries by heart would have helped.

I am not a heavy user of social media. I deleted my Facebook account several
years ago and communicate with friends via chat or email only. I don't talk
much, so no one missed me during the 48-hour period. For my senior design
project, which is still ongoing, my collaborators contacted me by text rather
than by email, but generally, not having Internet access made it easy for me to
hide.

It was most rewarding to be alone with my thoughts for hours rather than
distracting it with a continuous stream of news. I could be a lot more
productive if I learn to use the Internet only when I need it. Being constantly
connected makes it cheap to keep ``up-to-date'' with things, but that is
clearly unnecessary.

\end{document}
