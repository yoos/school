\documentclass[11pt,letterpaper]{article}
\usepackage{amsmath,amssymb,xfrac}
\usepackage[margin=1in]{geometry}
\usepackage{fancyhdr}
\usepackage[utf8]{inputenc}
\usepackage{palatino}
\usepackage{microtype}
\usepackage{hyperref}
\usepackage{graphicx}
\usepackage{lastpage}
\usepackage[hang,small,margin=1in]{caption}
\usepackage{titlesec}

\renewcommand{\headrulewidth}{0pt}
\fancyfoot{}
\fancyfoot[C]{\sffamily Page \thepage\ of~\pageref{LastPage}}
\pagestyle{fancy}

\titleformat{\section}{\bfseries\MakeUppercase}{\arabic{\thesection}}{1em}{}
\titleformat{\subsection}{\bfseries}{\arabic{\thesection}.\arabic{\thesubsection}}{1em}{}
\titleformat{\subsubsection}{\itshape}{\arabic{\thesection}.\arabic{\thesubsection}.\arabic{\thesubsubsection}}{1em}{}

\setlength{\parindent}{0cm}
\setlength{\parskip}{1em}

\captionsetup[figure]{labelfont=it, font=it}
\captionsetup[table]{labelfont={it,sc}, font={it,sc}}

\hypersetup{colorlinks, linkcolor = black, citecolor = black, urlcolor = black}
\urlstyle{same}



\begin{document}

\fancyfoot{}
\begin{center}
  \hfill \\
  \vspace{4in}
  {\bf\Huge MTH351 Assignment 6} \\
  \vspace{2in}
  {\Large Soo-Hyun Yoo \\ 930569466 \\ June 5, 2015}
\end{center}

\newpage
\fancyfoot[C]{\sffamily Page \thepage\ of~\pageref{LastPage}}

% Use notes from weeks 8 and 9

\begin{enumerate}
  \item
    \begin{enumerate}
      \item
        \begin{align*}
          L_0(x) &= \frac{(x-a-h)(x-a-2h)(x-a-3h)}{(a-a-h)(a-a-2h)(a-a-3h)} \\
                 &= \frac{(x-a-h)(x-a-2h)(x-a-3h)}{-6h^3} \\
          w_0 &= \int_a^bL_0(x)\;dx
        \end{align*}
        We substitute $u = x-a$:
        \begin{align*}
          \int_a^bL_0(x)\;dx &= \int_0^{b-a} \frac{(u-h)(u-2h)(u-3h)}{-6h^3}\;du \\
                             &= -\frac{1}{6h^3} \int_0^{b-a}u^3-6hu^2+11h^2u-6h^3\;du \\
                             &= -\frac{1}{6h^3} \left[\frac14(x-a)^4 - 2h(x-a)^3 + \frac{11}{2}h^2(x-a)^2 - 6h^3(x-a)\right]_a^b \\
                             &= -\frac{1}{6\left(\frac{b-a}{3}\right)^3} \left(\frac{(b-a)^4}{4} - 2\frac{(b-a)^4}{3} + \frac{11}{2}\cdot\frac{(b-a)^4}{3^2} - 6\frac{(b-a)^4}{3^3}\right) \\
                             &= -\frac{3^3}{6(b-a)^3} \cdot \frac{(9-24+22-8)(b-a)^4}{36} \\
                             &= \frac{27(b-a)^4}{6(b-a)^3\cdot36} \\
                             &= \frac{b-a}{8} = \frac{3h}{8} \quad \checkmark
        \end{align*}
        Similarly, we have:
        \begin{align*}
          L_1(x) &= \frac{(x-a)(x-a-2h)(x-a-3h)}{(a+h-a)(a+h-a-2h)(a+h-a-3h)} \\
                 &= \frac{(x-a)(x-a-2h)(x-a-3h)}{2h^3} \\
          w_1 &= \int_a^bL_1(x)\;dx
        \end{align*}
        Substituting $u = x-a$ again,
        \begin{align*}
          \int_a^bL_1(x)\;dx &= \int_0^{b-a} \frac{(u)(u-2h)(u-3h)}{2h^3}\;du \\
                             &= \frac{1}{2h^3} \int_0^{b-a}u^3-5hu^2+6h^2u\;du \\
                             &= \frac{3^3}{2(b-a)^3} \left(\frac{(b-a)^4}{4} - \frac53\cdot\frac{(b-a)^4}{3} + 3\frac{(b-a)^4}{3^2}\right) \\
                             &= \frac{3^3(b-a)^4}{2(b-a)^3\cdot36} \\
                             &= \frac{3(b-a)}{8} = \frac{9h}{8} \quad \checkmark
        \end{align*}

      \item The true value of the integral $\int_0^3\frac{1}{x+1}\;dx = 1.3863$.
        Approximation with Simpson's rule:
        \[\frac{3-0}{6}\left(\frac{1}{0+1} + \frac{4}{\frac32+1} + \frac{1}{3+1}\right) = 1.425\]
        Approximation with Simpson's three-eighths rule:
        \[\frac{3-0}{8}\left(\frac{1}{0+1} + \frac{3}{1+1} + \frac{3}{2+1} + \frac{1}{3+1}\right) = 1.4062\]
        The error using Simpson's rule is $\frac{1.425-1.3863}{1.4062-1.3863}
        = 1.9447$ times as large as that when using Simpson's three-eighths
        rule.
    \end{enumerate}

  \item
    \begin{enumerate}
      \item
        {\footnotesize
        \begin{verbatim}
function I = comp_simp(x,y)
    h = x(2)-x(1);   % assume equal spacing
    I = y(x(1)) + 4*sum(y(x(2:2:end-1))) + 2*sum(y(x(3:2:end-2))) + y(x(end));
    I = h*I/3;
end
        \end{verbatim}}

      \item Integrating $\int_0^3\frac{1}{x+1}\;dx$ with spacing $h=0.25$ using
        the code above returns $1.3864$, which is within $0.0001$ of the true
        value.

      \item Maximum $f^{(4)}(\xi) = \frac{24}{(x+1)^5}$ on $[0,3]$ occurs when
        x is smallest, at $x=0$, and is 24. This places an upper bound on the
        error $E_{12}^S(f)
        = \frac{-(3-0)\cdot0.25^4}{180}\cdot24
        = \boxed{0.0015625}$, which is about 15 times larger than the actual
        error found in part (b).

      \item The asymptotic error estimate $\overset{\sim}{E_{12}^S}(f)
        = \frac{-0.25^4}{180}\left(-\frac{6}{(3+1)^4}+\frac{6}{(0+1)^4}\right)
        = \boxed{0.00012970}$, which is on the same order of magnitude as the
        actual error.
    \end{enumerate}

  \item
    \begin{enumerate}
      \item From symmetry we have
        \begin{align*}
          w_0 &= w_2 \\
          x_1 &= 0,
        \end{align*}
        so
        \begin{align*}
          &\int_{-1}^11\;dx   = 2 = 2w_0 + w_1 \\
          &\int_{-1}^1x^2\;dx = \frac23 = 2w_0x_0^2 \\
          &\int_{-1}^1x^4\;dx = \frac25 = 2w_0x_0^4.
        \end{align*}
        From the last two integrals we have
        \begin{align*}
          \frac{2w_0x_0^4}{2w_0x_0^2} &= \frac{2/5}{2/3} \\
          x_0^2 &= \frac35 \\
          x_0 &= -\sqrt{\frac35}
        \end{align*}
        and
        \begin{align*}
          2w_0x_0^2 &= \frac23 \\
          w_0 &= \frac23 \cdot \frac53 \cdot \frac12 \\
              &= \frac59,
        \end{align*}
        which yields:
        \begin{align*}
          \begin{matrix}
            w_0 = \cfrac59         & w_1 = \cfrac89 & w_2 = \cfrac59 \\
            x_0 = -\sqrt{\cfrac35} & x_1 = 0        & x_2 = \sqrt{\cfrac35}
          \end{matrix}
        \end{align*}

      \item The true value of the integral $\int_{-1}^1\frac{1}{4+x^2}\;dx = \arctan(\frac12) = 0.46365$.
        The Simpson's rule approximation is \[\frac13\left(\frac15+1+\frac15\right) = 0.46667.\]
        The quadrature rule approximation is \[\frac59\cdot\frac{1}{4+\frac35} + \frac89\cdot\frac14 + \frac59\cdot\frac{1}{4+\frac35} = 0.46377.\]
        The error is 0.00302 and 0.00012 for Simpson's rule and the quadrature
        rule, respectively. The Simpson's rule error is about 1.5 orders of
        magnitude greater than the quadrature rule error.
    \end{enumerate}
\end{enumerate}

\end{document}
