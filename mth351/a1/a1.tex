\documentclass[12pt,letterpaper]{article}
\usepackage{amsmath}
\usepackage[margin=1in]{geometry}
\usepackage{fancyhdr}
\usepackage[utf8]{inputenc}
\usepackage{palatino}
\usepackage{microtype}
\usepackage{hyperref}
\usepackage{graphicx}
\usepackage{lastpage}
\usepackage[hang,small,margin=1in]{caption}
\usepackage{titlesec}

\renewcommand{\headrulewidth}{0pt}
\fancyfoot{}
\fancyfoot[C]{\sffamily Page \thepage\ of~\pageref{LastPage}}
\pagestyle{fancy}

\titleformat{\section}{\bfseries\MakeUppercase}{\arabic{\thesection}}{1em}{}
\titleformat{\subsection}{\bfseries}{\arabic{\thesection}.\arabic{\thesubsection}}{1em}{}
\titleformat{\subsubsection}{\itshape}{\arabic{\thesection}.\arabic{\thesubsection}.\arabic{\thesubsubsection}}{1em}{}

\setlength{\parindent}{0cm}
\setlength{\parskip}{1em}

\captionsetup[figure]{labelfont=it, font=it}
\captionsetup[table]{labelfont={it,sc}, font={it,sc}}

\hypersetup{colorlinks, linkcolor = black, citecolor = black, urlcolor = black}
\urlstyle{same}



\begin{document}

\fancyfoot{}
\begin{center}
  \hfill \\
  \vspace{4in}
  {\bf\Huge MTH351 Assignment 1} \\
  \vspace{2in}
  {\Large Soo-Hyun Yoo \\ 930569466 \\ April 3, 2015}
\end{center}

\newpage
\fancyfoot[C]{\sffamily Page \thepage\ of~\pageref{LastPage}}

\begin{enumerate}
  \item
    \begin{enumerate}
      \item 16384. When adding 10000 to x, the value of 10000 cannot be
        represented with enough significant digits in the same significand used
        to represent 1e20. This results in a loss of significance.
      \item {\tt 1/(eps(1)/2) = 9.0072e+15}
    \end{enumerate}
  \item
    \begin{enumerate}
      \item Large $x$ will result in high cancellation error. For
        $x={\tt single(1e6)}$, $\sqrt{1+\frac{1}{x}}-1 = {\tt 4.7684e-07}$. An
        equivalent expression $\cfrac{1}{x\left(\sqrt{1+\frac{1}{x}}+1\right)}
        = {\tt
        5.0000e-07}$.
      \item $x$ close to zero will result in high cancellation error. For
        $x={\tt single(1e-6)}$, $\frac{e^x-e^{-x}}{x} = 1.9670$. The Taylor
        series approximation $2+\frac{x^2}{3}+\frac{x^4}{60} = 2$.
    \end{enumerate}
  \item
    \begin{enumerate}
      \item The roots turn from $-2, -8$ to $-1.9997, -8.0003$.
      \item The roots turn from $-4, -4$ to $-3.96, -4.04$.
      \item Problem a is well-conditioned, as a change in $\gamma$ results in
        a reasonably proportional change in the roots. Problem b is
        ill-conditioned due to the closeness of $\beta$ to $\gamma$, resulting
        in large changes in the roots for a small change in $\gamma$.
    \end{enumerate}
  \item
    \begin{enumerate}
      \item As $x_1$ is a root of $g$, $g(x_1)$ should be zero. In this case,
        however, $g(x_1) = -0.02502441$, which is inaccurate considering the
        number of digits in $x_1$. It is the calculation of $x$ that is causing
        the inaccuracy.
      \item[(b,c)] $x = \cfrac{-2q}{u^2+uv+v^2} = {\tt -8.33333121e-04}$ gives
        $g(x) = 0.00000286$, which is much closer to the ideal value of $0$.
        This new expression for $x$ avoids the cancellation error-prone
        operation of $u-v$, where $u$ and $v$ are very close in value.
    \end{enumerate}
\end{enumerate}

\end{document}
