\documentclass[12pt,letterpaper]{article}
\usepackage{amsmath,amssymb,xfrac}
\usepackage[margin=1in]{geometry}
\usepackage{fancyhdr}
\usepackage[utf8]{inputenc}
\usepackage{palatino}
\usepackage{microtype}
\usepackage{hyperref}
\usepackage{graphicx}
\usepackage{lastpage}
\usepackage[hang,small,margin=1in]{caption}
\usepackage{titlesec}

\renewcommand{\headrulewidth}{0pt}
\fancyfoot{}
\fancyfoot[C]{\sffamily Page \thepage\ of~\pageref{LastPage}}
\pagestyle{fancy}

\titleformat{\section}{\bfseries\MakeUppercase}{\arabic{\thesection}}{1em}{}
\titleformat{\subsection}{\bfseries}{\arabic{\thesection}.\arabic{\thesubsection}}{1em}{}
\titleformat{\subsubsection}{\itshape}{\arabic{\thesection}.\arabic{\thesubsection}.\arabic{\thesubsubsection}}{1em}{}

\setlength{\parindent}{0cm}
\setlength{\parskip}{1em}

\captionsetup[figure]{labelfont=it, font=it}
\captionsetup[table]{labelfont={it,sc}, font={it,sc}}

\hypersetup{colorlinks, linkcolor = black, citecolor = black, urlcolor = black}
\urlstyle{same}



\begin{document}

\fancyfoot{}
\begin{center}
  \hfill \\
  \vspace{4in}
  {\bf\Huge MTH351 Assignment 3} \\
  \vspace{2in}
  {\Large Soo-Hyun Yoo \\ 930569466 \\ May 1, 2015}
\end{center}

\newpage
\fancyfoot[C]{\sffamily Page \thepage\ of~\pageref{LastPage}}

\begin{enumerate}
  \item
    \begin{enumerate}
      \item If $\phi$ is a fixed point of $g(x)$, then $g(\phi) = \phi$. Since
        $\phi$ is a positive root of $f(x)$, to show that $\phi$ is a fixed
        point of the three functions, it is sufficient to show that such
        a point is also a root of $f(x)$.
        \begin{align*}
          g_1(\phi) = \phi &= \sqrt{\phi+1} \\
                    \phi^2 &= \phi+1 \\
                         0 &= \phi^2 - \phi - 1 = f(\phi) \quad \checkmark
        \end{align*}
        \begin{align*}
          g_2(\phi) = \phi &= \phi^2-1 \\
                         0 &= \phi^2 - \phi - 1 = f(\phi) \quad \checkmark
        \end{align*}
        \begin{align*}
          g_3(\phi) = \phi &= \frac{\phi^2+1}{2\phi-1} \\
              2\phi^2-\phi &= \phi^2+1 \\
                         0 &= \phi^2 - \phi - 1 = f(\phi) \quad \checkmark
        \end{align*}
      \item Per the Contraction Mapping Theorem, if $g'(x)\neq0$ at a fixed
        point $x$, the convergence of $g$ at $x$ will be linear; otherwise it
        will be quadratic.
        \begin{enumerate}
          \item $g_1'(x) = \frac{1}{2\sqrt{x+1}}$. $g_1'(\phi) \neq 0$ and is
            clearly less than 1 near $\phi$, so it will converge linearly.
          \item $g_2'(x) = 2x$. This exceeds 1 near $\phi$, so it diverges.
          \item $g_3'(x) = \frac{2(x^2-x-1)}{(2x-1)^2}$. $g_3'(\phi) = 0$ and
            is reasonably small near $\phi$, so it will converge quadratically.
        \end{enumerate}
      \item
        \begin{enumerate}
          \item $g_1$ converges linearly after 10 iterations to
            1.618030905177271, which is accurate to 5 decimal digits. It seems
            to slowly converge, as predicted.
          \item $g_2$ diverges. After 10 iterations, the value is actually
            negative.
          \item $g_3$ converges quadratically after only 5 iterations to all 16
            digits.
        \end{enumerate}
    \end{enumerate}

  \item First, we find $L$ and $U$:
    \begin{align*}
      A =
      \begin{bmatrix}
        1 & 2 & 4 \\
        2 & 7 & 14 \\
        4 & 14 & 30
      \end{bmatrix}
      &\rightarrow
      \begin{bmatrix}
        1 & 2 & 4 \\
        0 & 3 & 6 \\
        0 & 6 & 14
      \end{bmatrix}
      \quad
      \begin{matrix}
        \\
        R_2 \leftarrow R_2 - 2R_1 \\
        R_3 \leftarrow R_3 - 4R_1
      \end{matrix} \\
      &\rightarrow
      \begin{bmatrix}
        1 & 2 & 4 \\
        0 & 3 & 6 \\
        0 & 0 & 2
      \end{bmatrix}
      \quad
      \begin{matrix}
        \\
        \\
        R_3 \leftarrow R_3 - 2R_1
      \end{matrix} \\
      &= U
    \end{align*}
    So
    \begin{align*}
      L =
      \begin{bmatrix}
        1 & 0 & 0 \\
        2 & 1 & 0 \\
        4 & 2 & 1
      \end{bmatrix}.
    \end{align*}
    Since $A=LU$, we have $LUx=b$. Let $y=Ux$. Then we have $Ly=b$, so we have
    \begin{align*}
      \begin{bmatrix}
        1 & 0 & 0 \\
        2 & 1 & 0 \\
        4 & 2 & 1
      \end{bmatrix}
      \begin{bmatrix}
        y_1 \\ y_2 \\ y_3
      \end{bmatrix}
      &=
      \begin{bmatrix}
        3 \\ 6 \\ 10
      \end{bmatrix}.
    \end{align*}
    Solving forward, we have:
    \begin{align*}
      y_1 &= 3 \\
      y_2 &= 6-2(3) = 0 \\
      y_3 &= 10-4(3)-2(0) = -2
    \end{align*}
    Finally, we have $Ux=y$:
    \begin{align*}
      \begin{bmatrix}
        1 & 2 & 4 \\
        0 & 3 & 6 \\
        0 & 0 & 2
      \end{bmatrix}
      \begin{bmatrix}
        x_1 \\ x_2 \\ x_3
      \end{bmatrix}
      &=
      \begin{bmatrix}
        3 \\ 0 \\ -2
      \end{bmatrix}
    \end{align*}
    Solving backwards, we have:
    \begin{align*}
      x_3 &= \frac{-2}{2} = -1 \\
      x_2 &= \frac{0-6(-1)}{3} = 2 \\
      x_1 &= \frac{3-2(2)-4(-1)}{1} = 3
    \end{align*}
    That is,
    \begin{align*}
      x =
      \begin{bmatrix}
        3 \\ 2 \\ -1
      \end{bmatrix}.
    \end{align*}

  \item
    \begin{enumerate}
      \item If $U=D\tilde{U}$, \[u_{ij} = \sum_{k=1}^n{d_{ik}\tilde{u}_{kj}}.\]
        Since $D$ is diagonal, $d_{ij} = 0$ for $i \neq j$, so \[u_{ij}
        = d_{ii}\tilde{u}_{ij}.\] That is, each $i$th row in $U$ is the $i$th
        row in $\tilde{U}$ scaled by $d_{ii}$.

      \item Since $D$ is diagonal, $D=D^T$. If $A=A^T$, then we have:
        \begin{align*}
          LD\tilde{U} &= \tilde{U}^TD^TL^T \\
          LD\tilde{U} &= \tilde{U}^TDL^T
        \end{align*}
        So $\tilde{U}^T=L \equiv \tilde{U}=L^T$, which means
        $A=LD\tilde{U}=LDL^T$.

      \item If we can express $D$ as $D^{\frac12}D^{\frac12}$, then
        \begin{align*}
          A &= LD\tilde{U} \\
            &= LD^{\frac12}D^{\frac12}\tilde{U} \\
            &= LD\left(\tilde{U}^TD^{\frac12T}\right)^T \\
            &= LD\left(LD\right)^T \\
            &= \tilde{L}\tilde{L}^T,
        \end{align*}
        where $\tilde{L}=LD$.

      \item From $U$, we can find
        \begin{align*}
          D &=
          \begin{bmatrix}
            1 & 0 & 0 \\
            0 & 3 & 0 \\
            0 & 0 & 2
          \end{bmatrix} \\
          D^{\frac12} &=
          \begin{bmatrix}
            1 & 0 & 0 \\
            0 & \sqrt{3} & 0 \\
            0 & 0 & \sqrt{2}
          \end{bmatrix}.
        \end{align*}
        Then
        \begin{align*}
          \tilde{L} = LD^{\frac12} &=
          \begin{bmatrix}
            1 & 0 & 0 \\
            2 & 1 & 0 \\
            4 & 2 & 1
          \end{bmatrix}
          \begin{bmatrix}
            1 & 0 & 0 \\
            0 & \sqrt{3} & 0 \\
            0 & 0 & \sqrt{2}
          \end{bmatrix} \\
          &=
          \begin{bmatrix}
            1 & 0 & 0 \\
            2 & 1\sqrt{3} & 0 \\
            4 & 2\sqrt{3} & 1\sqrt{2}
          \end{bmatrix},
        \end{align*}
        so the Cholesky factorization of A is as follows:
        \begin{align*}
          A &=
          \begin{bmatrix}
            1 & 0 & 0 \\
            2 & 1\sqrt{3} & 0 \\
            4 & 2\sqrt{3} & 1\sqrt{2}
          \end{bmatrix}
          \begin{bmatrix}
            1 & 2 & 4 \\
            0 & 1\sqrt{3} & 2\sqrt{3} \\
            0 & 0 & 1\sqrt{2}
          \end{bmatrix}
        \end{align*}
    \end{enumerate}
\end{enumerate}

\end{document}
