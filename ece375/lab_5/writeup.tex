% template created by: Russell Haering. arr. Joseph Crop
\documentclass[12pt,letterpaper]{article}
\usepackage{anysize}
\marginsize{2cm}{2cm}{1cm}{1cm}
\usepackage{listings}
\usepackage{color}

\definecolor{dkgreen}{rgb}{0,0.6,0}
\definecolor{gray}{rgb}{0.5,0.5,0.5}
\definecolor{mauve}{rgb}{0.58,0,0.82}

\lstset{
	language={[x86masm]Assembler},
	basicstyle=\footnotesize,           % the size of the fonts that are used for the code
	numbers=left,                   % where to put the line-numbers
	numberstyle=\tiny\color{gray},  % the style that is used for the line-numbers
	stepnumber=1,                   % the step between two line-numbers. If it's 1, each line~
	                                % will be numbered
	numbersep=5pt,                  % how far the line-numbers are from the code
	backgroundcolor=\color{white},      % choose the background color. You must add \usepackage{color}
	showspaces=false,               % show spaces adding particular underscores
	showstringspaces=false,         % underline spaces within strings
	showtabs=false,                 % show tabs within strings adding particular underscores
	frame=single,                   % adds a frame around the code
	rulecolor=\color{black},        % if not set, the frame-color may be changed on line-breaks within not-black text (e.g. commens (green here))
	tabsize=2,                      % sets default tabsize to 2 spaces
	captionpos=b,                   % sets the caption-position to bottom
	breaklines=true,                % sets automatic line breaking
	breakatwhitespace=false,        % sets if automatic breaks should only happen at whitespace  
	title=\lstname,                   % show the filename of files included with \lstinputlisting;
	                                % also try caption instead of title
	keywordstyle=\color{blue},          % keyword style
	commentstyle=\color{dkgreen},       % comment style
	stringstyle=\color{mauve},         % string literal style
	escapeinside={\%*}{*)},            % if you want to add LaTeX within your code
	morekeywords={*,...}               % if you want to add more keywords to the set
}

\begin{document}

\begin{titlepage}
    \vspace*{4cm}
    \begin{flushright}
    {\huge
        ECE 375 Lab 5\\[1cm]
    }
    {\large
        Remotely Operated Vehicle
    }
    \end{flushright}
    \begin{flushleft}
    Lab Time: Wednesday 5-7
    \end{flushleft}
    \begin{flushright}
    Soo-Hyun Yoo

    \vfill
    \rule{5in}{.5mm}\\
    TA Signature
    \end{flushright}

\end{titlepage}

\section*{Introduction}

% Introduction: purpose of lab

\section*{Program Overview}

% Program Overview: detailed description of assembly program
In the two assembly files, tx.asm and rx.asm, USART1's transmit and receive
capabilities (respectively) are enabled by setting TXEN1 and RXEN1 high in
UBSR1B.

The TX code checks for button presses by polling all buttons in the main
functions and matching them (with a lot of brne's and rjmp's) to the known
button configurations. If any one of the buttons are pressed, a transmit
function is called, which first transmits the device ID and the action code.
There is a wait function at the end of the transmit function to avoid spamming
the receiver with too many commands. For debugging purposes, the commands sent
are also displayed using the red LEDs, and the TX code is capable of sending
the freeze signal itself.

The RX code relies on interrupts for detecting whisker hits and for the USART
receive. Upon receiving, an RX function first disables interrupts and checks
whether it was the freeze command. If so, it freezes and decrements a freeze
counter that is initialized at 7 (for some reason, this number works instead of
3 in order for the RX node to freeze three times). If the freeze counter
reaches 0, it freezes permanently. Otherwise, it times out after 5 seconds.

If the command is not a freeze command, it checks if it matches the device ID.
If so, it waits until the next byte is received and matches it against the
known commands. At the end, the function reenables external interrupts.

Both TX and RX is performed after checking that the transmit and receive
buffers, respectively, are empty.


\section*{Additional Questions}

% Additional Questions: answers to questions in lab handout
No additional questions asked.


\section*{Difficulties}

% Difficulties: (optional but recommended) a description of any difficulties encountered
Due to absence, I was unable to work with a partner and implemented both TX and
RX code using a second TekBots board (not a TekBot).

The RX currently freezes itself when emitting the freeze signal, which suggests
that my disabling of global interrupts using cli is not actually working.
Setting the RXEN1 and RXCIE1 bits in UCSR1B to 0 did not help, either.


\section*{Conclusion}

% Conclusion: summary of lab with a personal thought
Implementing USART RX/TX was simple once I found the appropriate pages in the
datasheet. Keeping track of my stack and branches was tricky but reasonable.


\section*{Source Code}

% Source Code: assembly code MUST be included in your report and be well-documented.
\lstinputlisting{tx.asm}
\lstinputlisting{rx.asm}

\end{document}

