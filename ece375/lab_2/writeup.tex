% template created by: Russell Haering. arr. Joseph Crop
\documentclass[12pt,letterpaper]{article}
\usepackage{anysize}
\marginsize{2cm}{2cm}{1cm}{1cm}

\begin{document}

\begin{titlepage}
    \vspace*{4cm}
    \begin{flushright}
    {\huge
        ECE 375 Lab 2\\[1cm]
    }
    {\large
        Data Manipulation and the LCD
    }
    \end{flushright}
    \begin{flushleft}
    Lab Time: Wednesday 5-7
    \end{flushleft}
    \begin{flushright}
    Soo-Hyun Yoo

    \vfill
    \rule{5in}{.5mm}\\
    TA Signature
    \end{flushright}

\end{titlepage}

\section{Pre-lab}

\begin{itemize}
	\item The stack pointer is two bytes long and points to the top of a LIFO
		stack. SPL is the low byte and SPH is the high byte. Because stacks in
		AVR grow from higher to lower address space in memory, the stack
		pointer is initialized to the highest data address space available
		(RAMEND) in the INIT function. Thus, its initialization looks something
		like the following:

		\begin{verbatim}
1. SPH <- HIGH_BYTE(RAMEND)
2. SPL <- LOW_BYTE(RAMEND)
		\end{verbatim}

	\item LPM stands for ``Load Program Memory'' and serves to load data from
		the low byte of a Program Memory address (16-bit) into Data Memory
		(8-bit). LPM will:

		\begin{verbatim}
1. Left-shift value by 1 bit to account for selecting low/high byte.
2. Copy value
		\end{verbatim}

	\item m128def.inc is a machine-generated file. All I/O register and
		register bit names that appear in the datasheet are specified here
		(e.g., ADC, SPI, I2C, UART, interrupts, PIN, PORT). The file is
		included with the line, `.include ``m128def.inc'' '.

\end{itemize}

\end{document}
