% template created by: Russell Haering. arr. Joseph Crop
\documentclass[12pt,letterpaper]{article}
\usepackage{anysize}
\marginsize{2cm}{2cm}{1cm}{1cm}
\usepackage{listings}
\usepackage{color}

\definecolor{dkgreen}{rgb}{0,0.6,0}
\definecolor{gray}{rgb}{0.5,0.5,0.5}
\definecolor{mauve}{rgb}{0.58,0,0.82}

\lstset{
	language={[x86masm]Assembler},
	basicstyle=\footnotesize,           % the size of the fonts that are used for the code
	numbers=left,                   % where to put the line-numbers
	numberstyle=\tiny\color{gray},  % the style that is used for the line-numbers
	stepnumber=1,                   % the step between two line-numbers. If it's 1, each line~
	                                % will be numbered
	numbersep=5pt,                  % how far the line-numbers are from the code
	backgroundcolor=\color{white},      % choose the background color. You must add \usepackage{color}
	showspaces=false,               % show spaces adding particular underscores
	showstringspaces=false,         % underline spaces within strings
	showtabs=false,                 % show tabs within strings adding particular underscores
	frame=single,                   % adds a frame around the code
	rulecolor=\color{black},        % if not set, the frame-color may be changed on line-breaks within not-black text (e.g. commens (green here))
	tabsize=2,                      % sets default tabsize to 2 spaces
	captionpos=b,                   % sets the caption-position to bottom
	breaklines=true,                % sets automatic line breaking
	breakatwhitespace=false,        % sets if automatic breaks should only happen at whitespace  
	title=\lstname,                   % show the filename of files included with \lstinputlisting;
	                                % also try caption instead of title
	keywordstyle=\color{blue},          % keyword style
	commentstyle=\color{dkgreen},       % comment style
	stringstyle=\color{mauve},         % string literal style
	escapeinside={\%*}{*)},            % if you want to add LaTeX within your code
	morekeywords={*,...}               % if you want to add more keywords to the set
}

\begin{document}

\begin{titlepage}
    \vspace*{4cm}
    \begin{flushright}
    {\huge
        ECE 375 Lab 2\\[1cm]
    }
    {\large
        Data Manipulation and the LCD
    }
    \end{flushright}
    \begin{flushleft}
    Lab Time: Wednesday 5-7
    \end{flushleft}
    \begin{flushright}
    Soo-Hyun Yoo

    \vfill
    \rule{5in}{.5mm}\\
    TA Signature
    \end{flushright}

\end{titlepage}

\section{Pre-lab}

\begin{itemize}
	\item The stack pointer is two bytes long and points to the top of a LIFO
		stack. SPL is the low byte and SPH is the high byte. Because stacks in
		AVR grow from higher to lower address space in memory, the stack
		pointer is initialized to the highest data address space available
		(RAMEND) in the INIT function. Thus, its initialization looks something
		like the following:

		\begin{verbatim}
1. SPH <- HIGH_BYTE(RAMEND)
2. SPL <- LOW_BYTE(RAMEND)
		\end{verbatim}

	\item LPM stands for ``Load Program Memory'' and serves to load data from
		the low byte of a Program Memory address (16-bit) into Data Memory
		(8-bit). LPM will:

		\begin{verbatim}
1. Left-shift value by 1 bit to account for selecting low/high byte.
2. Copy value
		\end{verbatim}

	\item m128def.inc is a machine-generated file. All I/O register and
		register bit names that appear in the datasheet are specified here
		(e.g., ADC, SPI, I2C, UART, interrupts, PIN, PORT). The file is
		included with the line, `.include ``m128def.inc'' '.

\end{itemize}


\section{Writeup}

\begin{enumerate}
	\item The DEF directive is almost like a pointer in that it can be
		redefined to refer to a different register and that a register can be
		referred to by multiple symbols defined by DEF.

		EQU, on the other hand, cannot be redefined and is thus much like
		a C macro definition in usage.

	\item The Program Memory is non-volatile, so this is where the program (and
		constants) is stored. Because AVR instructions are 16 or 32 bits wide,
		the AVR Program Memory is divided into 16-bit chunks.

		The Data Memory is used to store variables and the addresses of
		registers, I/O Memory, and the internal data SRAM.

	\item When a function is called, the program stores its current location in
		the memory stack and jumps to the location of the function. Once the
		function is complete, the RET instruction brings the function back to
		its previous location using the stored information in the stack. If the
		stack pointer is not initialized, however, the program cannot do so.
		
		When the stack pointer initialization was removed from my code, the LCD
		did not show anything, which indicates that the program got stuck
		somewhere---most likely after the first RCALL of LCDInit.
\end{enumerate}

\newpage
\lstinputlisting{myLCD.asm}

\end{document}
