% template created by: Russell Haering. arr. Joseph Crop
\documentclass[12pt,letterpaper]{article}
\usepackage{anysize}
\marginsize{2cm}{2cm}{1cm}{1cm}
\usepackage{listings}
\usepackage{color}

\definecolor{dkgreen}{rgb}{0,0.6,0}
\definecolor{gray}{rgb}{0.5,0.5,0.5}
\definecolor{mauve}{rgb}{0.58,0,0.82}

\lstset{
	language={[x86masm]Assembler},
	basicstyle=\footnotesize,           % the size of the fonts that are used for the code
	numbers=left,                   % where to put the line-numbers
	numberstyle=\tiny\color{gray},  % the style that is used for the line-numbers
	stepnumber=1,                   % the step between two line-numbers. If it's 1, each line~
	                                % will be numbered
	numbersep=5pt,                  % how far the line-numbers are from the code
	backgroundcolor=\color{white},      % choose the background color. You must add \usepackage{color}
	showspaces=false,               % show spaces adding particular underscores
	showstringspaces=false,         % underline spaces within strings
	showtabs=false,                 % show tabs within strings adding particular underscores
	frame=single,                   % adds a frame around the code
	rulecolor=\color{black},        % if not set, the frame-color may be changed on line-breaks within not-black text (e.g. commens (green here))
	tabsize=2,                      % sets default tabsize to 2 spaces
	captionpos=b,                   % sets the caption-position to bottom
	breaklines=true,                % sets automatic line breaking
	breakatwhitespace=false,        % sets if automatic breaks should only happen at whitespace  
	title=\lstname,                   % show the filename of files included with \lstinputlisting;
	                                % also try caption instead of title
	keywordstyle=\color{blue},          % keyword style
	commentstyle=\color{dkgreen},       % comment style
	stringstyle=\color{mauve},         % string literal style
	escapeinside={\%*}{*)},            % if you want to add LaTeX within your code
	morekeywords={*,...}               % if you want to add more keywords to the set
}

\begin{document}

\begin{titlepage}
    \vspace*{4cm}
    \begin{flushright}
    {\huge
        ECE 375 Pre-Lab 5\\[1cm]
    }
    {\large
        Intel Atom Processor Lab
    }
    \end{flushright}
    \begin{flushleft}
    Lab Time: Wednesday 5-7
    \end{flushleft}
    \begin{flushright}
    Soo-Hyun Yoo

    \vfill
    \rule{5in}{.5mm}\\
    TA Signature
    \end{flushright}

\end{titlepage}

\section*{Pre-Lab}

\begin{enumerate}
	\item
		\begin{enumerate}
			\item Read a command byte from the I2C bus.
			\item Write a command byte to the I2C bus.
			\item Write a command byte followed by some data to the I2C bus.
		\end{enumerate}

	\item
		\begin{enumerate}
			\item setPen() is an overloaded function that sets the pen's typee,
				color, and style. drawLine() is an overloaded function that
				draws a line based on input points, tuples, or lines.
			\item setColor() sets the color of the current pen to the input
				color.
			\item setX() and setY() set the X and Y coordinates of a point to
				the given X and Y coordinates.
		\end{enumerate}

	\item The GPIO interface allows users to monitor or set the signal level on
		certain pins on the device. The SMBus is used for low-speed serial
		communication for system management. The I$^2$C bus is a bidirectional
		two-wire serial bus that is used ubiquitously today for communication
		by embedded devices.
\end{enumerate}

\end{document}

