\documentclass[12pt,letterpaper]{article}
\usepackage[margin=1in]{geometry}
\usepackage{fancyhdr}
\usepackage[utf8]{inputenc}
\usepackage{palatino}
\usepackage{microtype}
\usepackage{hyperref}
\usepackage{graphicx}
\usepackage{lastpage}
\usepackage[hang,bf,small]{caption}
\usepackage{titlesec}
\usepackage{amsmath,amssymb,amsthm}

\renewcommand{\headrulewidth}{0pt}
\fancyfoot{}
\fancyfoot[C]{\thepage}
\pagestyle{fancy}

% \titleformat{\section}{\bfseries\MakeUppercase}{\arabic{\thesection}}{1em}{}
% \titleformat{\subsection}{\bfseries}{\arabic{\thesection}.\arabic{\thesubsection}}{1em}{}
% \titleformat{\subsubsection}{\itshape}{\arabic{\thesection}.\arabic{\thesubsection}.\arabic{\thesubsubsection}}{1em}{}

\setlength{\parindent}{0cm}
\setlength{\parskip}{1em}

% \captionsetup[figure]{labelfont=it,font=it}
% \captionsetup[table]{labelfont={it,sc},font={it,sc}}

\hypersetup{colorlinks,
    linkcolor = black,
    citecolor = black,
    urlcolor  = black}
\urlstyle{same}


\begin{document}

Soo-Hyun Yoo \\
CS311 -- Homework 5 \\
19 November 2012


\section*{Design}

In order to run the Sieve of Eratosthenes in parallel, the main process will
fork off two processes. In order to find the number of primes up to $n$, the
first process will consider the first $\lceil\sqrt{n}\rceil$ numbers. The
second process will use a number of threads (specified on the command line) to
mark off composites that are passed from the first process.

% NOTE: The first process could actually be divided further. In that case, the
% very first process would behave the same, but the ``other'' first processes
% would have to act like the old second process in that it would mark off
% composites that they themselves did not find.


\section*{Worklog}


\section*{Challenges}


\section*{Answers to questions}


\end{document}
